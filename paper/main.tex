\documentclass[modern, letterpaper]{aastex61}

% to-do list
% ----------
% -

% style notes
% -----------
% - This file generates by Makefile; don't be typing ``pdflatex'' or some bullshit.
% - Line break between sentences to make the git diffs readable.
% - Simple Monte Carlo gets a capital S to indicate that it is a defined thing.
% - Use \, as a multiply operator.
% - Reserve () for function arguments; use [] or {} for outer shit.
% - Always prior pdf or posterior pdf, never prior or posterior (that's your arse).
% - Use \sectionname not Section, \figname not Figure, \documentname not Article or Paper or paper.

\include{gitstuff}
% Load common packages
% \usepackage{microtype}  % ALWAYS!
\usepackage{amsmath}
\usepackage{amsfonts}
\usepackage{amssymb}
\usepackage{booktabs}

\usepackage{graphicx}
\usepackage{color}

\definecolor{cbblue}{HTML}{3182bd}
\usepackage{hyperref}
\definecolor{linkcolor}{rgb}{0.02,0.35,0.55}
\definecolor{citecolor}{rgb}{0.45,0.45,0.45}
\hypersetup{colorlinks=true,linkcolor=linkcolor,citecolor=citecolor,
            filecolor=linkcolor,urlcolor=linkcolor}
\hypersetup{pageanchor=true}

\newcommand{\documentname}{\textsl{Article}}
\newcommand{\sectionname}{Section}
\renewcommand{\figurename}{Figure}
\newcommand{\eqname}{Equation}
\renewcommand{\tablename}{Table}

% Packages / projects / programming
\newcommand{\package}[1]{\textsl{#1}}
\newcommand{\acronym}[1]{{\small{#1}}}
\newcommand{\github}{\package{GitHub}}
\newcommand{\python}{\package{Python}}
\newcommand{\emcee}{\project{emcee}}

% Missions
\newcommand{\project}[1]{\textsl{#1}}

% For referee
\newcommand{\changes}[1]{{\color{red} #1}}

% Stats / probability
\newcommand{\given}{\,|\,}
\newcommand{\norm}{\mathcal{N}}

% Maths
\newcommand{\dd}{\mathrm{d}}
\newcommand{\transpose}[1]{{#1}^{\mathsf{T}}}
\newcommand{\inverse}[1]{{#1}^{-1}}
\newcommand{\argmin}{\operatornamewithlimits{argmin}}
\newcommand{\mean}[1]{\left< #1 \right>}

% Unit shortcuts
\newcommand{\msun}{\ensuremath{\mathrm{M}_\odot}}
\newcommand{\kms}{\ensuremath{\mathrm{km}~\mathrm{s}^{-1}}}
\newcommand{\mps}{\ensuremath{\mathrm{m}~\mathrm{s}^{-1}}}
\newcommand{\pc}{\ensuremath{\mathrm{pc}}}
\newcommand{\kpc}{\ensuremath{\mathrm{kpc}}}
\newcommand{\kmskpc}{\ensuremath{\mathrm{km}~\mathrm{s}^{-1}~\mathrm{kpc}^{-1}}}

% Misc.
\newcommand{\bs}[1]{\boldsymbol{#1}}
\definecolor{mahogany}{RGB}{165,15,21}
\newcommand{\resp}[1]{{\color{mahogany}#1}}

% Astronomy
\newcommand{\DM}{{\rm DM}}
\newcommand{\feh}{\ensuremath{{[{\rm Fe}/{\rm H}]}}}
\newcommand{\df}{\acronym{DF}}

% TO DO
\newcommand{\todo}[1]{{\color{red} TODO: #1}}


% define macros for text
\newcommand{\apogee}{\project{\acronym{APOGEE}}}
\newcommand{\sdssiii}{\project{\acronym{SDSS-III}}}
\newcommand{\thejoker}{\project{The~Joker}}
\newcommand{\DR}{\acronym{DR13}}

% for response to referee
% \renewcommand{\resp}[1]{#1}

\shortauthors{Price-Whelan et al.}

\begin{document}\sloppy\sloppypar\raggedbottom\frenchspacing % trust me

\title{Population statistics of companions to APOGEE red clump stars}

\author{Adrian~M.~Price-Whelan}
\affiliation{Department of Astrophysical Sciences,
             Princeton University, Princeton, NJ 08544, USA}
\email{adrn@astro.princeton.edu}
\correspondingauthor{Adrian M. Price-Whelan}

\author{David~W.~Hogg}
\affiliation{Center for Cosmology and Particle Physics,
             Department of Physics,
             New York University, 726 Broadway,
             New York, NY 10003, USA}
\affiliation{Max-Planck-Institut f\"ur Astronomie,
             K\"onigstuhl 17, D-69117 Heidelberg, Germany}
\affiliation{Flatiron Institute,
             Simons Foundation,
             162 Fifth Avenue,
             New York, NY 10010, USA}

\author{Hans-Walter~Rix}
\affiliation{Max-Planck-Institut f\"ur Astronomie,
             K\"onigstuhl 17, D-69117 Heidelberg, Germany}

\begin{abstract}
% Context
% Aims
% Methods
% Results
\end{abstract}

\keywords{
  binaries: spectroscopic
  ---
  methods: data analysis
  ---
  methods: statistical
  ---
  planets and satellites: fundamental parameters
  ---
  surveys
  ---
  techniques: radial velocities
}

\section{Introduction} \label{sec:intro}

Foo.

\end{document}
