\documentclass[modern, letterpaper]{aastex61}

% to-do list
% ----------
% -

% style notes
% -----------
% - This file generates by Makefile; don't be typing ``pdflatex'' or some bullshit.
% - Line break between sentences to make the git diffs readable.
% - Simple Monte Carlo gets a capital S to indicate that it is a defined thing.
% - Use \, as a multiply operator.
% - Reserve () for function arguments; use [] or {} for outer shit.
% - Always prior pdf or posterior pdf, never prior or posterior (that's your arse).
% - Use \sectionname not Section, \figname not Figure, \documentname not Article or Paper or paper.

\include{gitstuff}
% Load common packages
% \usepackage{microtype}  % ALWAYS!
\usepackage{amsmath}
\usepackage{amsfonts}
\usepackage{amssymb}
\usepackage{booktabs}

\usepackage{graphicx}
\usepackage{color}

\definecolor{cbblue}{HTML}{3182bd}
\usepackage{hyperref}
\definecolor{linkcolor}{rgb}{0.02,0.35,0.55}
\definecolor{citecolor}{rgb}{0.45,0.45,0.45}
\hypersetup{colorlinks=true,linkcolor=linkcolor,citecolor=citecolor,
            filecolor=linkcolor,urlcolor=linkcolor}
\hypersetup{pageanchor=true}

\newcommand{\documentname}{\textsl{Article}}
\newcommand{\sectionname}{Section}
\renewcommand{\figurename}{Figure}
\newcommand{\eqname}{Equation}
\renewcommand{\tablename}{Table}

% Packages / projects / programming
\newcommand{\package}[1]{\textsl{#1}}
\newcommand{\acronym}[1]{{\small{#1}}}
\newcommand{\github}{\package{GitHub}}
\newcommand{\python}{\package{Python}}
\newcommand{\emcee}{\project{emcee}}

% Missions
\newcommand{\project}[1]{\textsl{#1}}

% For referee
\newcommand{\changes}[1]{{\color{red} #1}}

% Stats / probability
\newcommand{\given}{\,|\,}
\newcommand{\norm}{\mathcal{N}}

% Maths
\newcommand{\dd}{\mathrm{d}}
\newcommand{\transpose}[1]{{#1}^{\mathsf{T}}}
\newcommand{\inverse}[1]{{#1}^{-1}}
\newcommand{\argmin}{\operatornamewithlimits{argmin}}
\newcommand{\mean}[1]{\left< #1 \right>}

% Unit shortcuts
\newcommand{\msun}{\ensuremath{\mathrm{M}_\odot}}
\newcommand{\kms}{\ensuremath{\mathrm{km}~\mathrm{s}^{-1}}}
\newcommand{\pc}{\ensuremath{\mathrm{pc}}}
\newcommand{\kpc}{\ensuremath{\mathrm{kpc}}}
\newcommand{\kmskpc}{\ensuremath{\mathrm{km}~\mathrm{s}^{-1}~\mathrm{kpc}^{-1}}}

% Misc.
\newcommand{\bs}[1]{\boldsymbol{#1}}
\definecolor{mahogany}{RGB}{165,15,21}
\newcommand{\resp}[1]{{\color{mahogany}#1}}

% Astronomy
\newcommand{\DM}{{\rm DM}}
\newcommand{\feh}{\ensuremath{{[{\rm Fe}/{\rm H}]}}}
\newcommand{\df}{\acronym{DF}}

% TO DO
\newcommand{\todo}[1]{{\color{red} TODO: #1}}


% define macros for text
\newcommand{\apogee}{\project{\acronym{APOGEE}}}
\newcommand{\sdssiii}{\project{\acronym{SDSS-III}}}
\newcommand{\thejoker}{\project{The~Joker}}
\newcommand{\thecannon}{\project{The~Cannon}}
\newcommand{\DR}{\acronym{DR13}}
\newcommand{\RC}{\acronym{RC}}

% for response to referee
% \renewcommand{\resp}[1]{#1}

\shortauthors{Price-Whelan et al.}

\begin{document}\sloppy\sloppypar\raggedbottom\frenchspacing % trust me

\title{Population statistics of companions to APOGEE red clump stars}

\author{Adrian~M.~Price-Whelan}
\affiliation{Department of Astrophysical Sciences,
             Princeton University, Princeton, NJ 08544, USA}
\email{adrn@astro.princeton.edu}
\correspondingauthor{Adrian M. Price-Whelan}

\author{David~W.~Hogg}
\affiliation{Center for Cosmology and Particle Physics,
             Department of Physics,
             New York University, 726 Broadway,
             New York, NY 10003, USA}
\affiliation{Max-Planck-Institut f\"ur Astronomie,
             K\"onigstuhl 17, D-69117 Heidelberg, Germany}
\affiliation{Flatiron Institute,
             Simons Foundation,
             162 Fifth Avenue,
             New York, NY 10010, USA}

\author{Hans-Walter~Rix}
\affiliation{Max-Planck-Institut f\"ur Astronomie,
             K\"onigstuhl 17, D-69117 Heidelberg, Germany}

\begin{abstract}
% Context
Radial-velocity measurements can be used to find stellar, sub-stellar,
and planetary companions of stars.
In the \apogee\ survey, we have multi-epoch radial-velocity measurements
for many red-giant stars, including many red-clump (\RC) stars.
Spectral analysis of these stars can be used to deliver masses (with
XXX precision) and 20-ish chemical abundance measurements (with YYY
precision).
% Aims
Here we perform a search for secondary companions of [number] \RC\ stars.
Because the primary-star masses are known, the mass-function degeneracy
is broken and the secondary-companion $m\,\sin i$ values can be inferred.
Also, if the primary and companion are assumed to be co-eval, the
abundances of the \RC\ star can be assumed to represent accurately the
abundances of the secondary.
% Methods
We use a data-driven model (\thecannon) of stellar spectra to deliver
each \RC\ mass and chemical abundances, and a custom-built Monte Carlo
sampler (\thejoker) to deliver (often highly multi-modal) posterior
beliefs about companion $m\,\sin i$, period, and other orbital parameters.
Our sampler permits us to accurately sample even the badly behaved
posterior pdfs obtained when there are as few an NNN observation
epochs per star.
We put these posterior samples---made under an interim prior---into a
hierarchical inference to obtain individual-companion-marginalized
posterior beliefs about the properties of the full secondary-companion
distribution in mass and period.
% Results
We deliver a catalog of QQQ companions [that meet some criteria], and
posterior samplings for all [number] stars in the parent sample.
We find that the period and mass distributions [have some properties].
We look for trends with chemical abundances, and we find tentative evidence
for ZZZ and WWW.
\end{abstract}

\keywords{
  binaries: spectroscopic
  ---
  methods: data analysis
  ---
  methods: statistical
  ---
  planets and satellites: fundamental parameters
  ---
  surveys
  ---
  techniques: radial velocities
}

\section{Introduction} \label{sec:intro}

Foo.

\end{document}
