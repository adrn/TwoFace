\begin{table}[htb]
    \footnotesize
    \centering
    \begin{tabular}{l|l|l}
        \hline
        Column name & Unit / format & Description \\
        \hline
        \texttt{APOGEE\_ID}        &                          &
            identifier used by \apogee \\
        \texttt{P}                 & $\mathrm{d}$             & $P$, period \\
        \texttt{P\_err}            & $\mathrm{d}$             & \\
        \texttt{M0}                & $\mathrm{rad}$           &
            $M_0$, phase at reference epoch \\
        \texttt{M0\_err}           & $\mathrm{rad}$           & \\
        \texttt{e}                 &                          &
            $e$, eccentricity \\
        \texttt{e\_err}            &                          & \\
        \texttt{omega}             & $\mathrm{rad}$           &
            $\omega$, argument of pericenter \\
        \texttt{omega\_err}        & $\mathrm{rad}$           & \\
        \texttt{jitter}            & $\mathrm{km\,s^{-1}}$    &
            $s$, excess variance parameter \\
        \texttt{jitter\_err}       & $\mathrm{km\,s^{-1}}$    & \\
        \texttt{K}                 & $\mathrm{km\,s^{-1}}$    &
            $K$, velocity semi-amplitude \\
        \texttt{K\_err}            & $\mathrm{km\,s^{-1}}$    & \\
        \texttt{v0}                & $\mathrm{km\,s^{-1}}$    &
            $v_0$, systemic velocity \\
        \texttt{v0\_err}           & $\mathrm{km\,s^{-1}}$    & \\
        \texttt{t0}                & Barycentric MJD          &
            $t_0$, reference epoch \\
        \texttt{converged}  &                          & binary flag
            indicating whether the sampling converged \\
        \texttt{Gelman-Rubin}      &                          &
            Gelman-Rubin statistic for each MCMC parameter \\
        \texttt{M1}                & $\mathrm{M_{\odot}}$     &
            primary mass estimate (\citealt{Ness:2015}) \\
        \texttt{M1\_err}           & $\mathrm{M_{\odot}}$     & \\
        \texttt{M2\_min}           & $\mathrm{M_{\odot}}$     &
            $M_{2, \textrm{min}}$, minimum $M_2$ mass \\
        \texttt{M2\_min\_err}      & $\mathrm{M_{\odot}}$     & \\
        \texttt{clean\_flag}       &                          &
            [0 = good, 1 = suspicious, 2 = bad], score from by-eye vetting \\
        \vdots & & all columns from \citet{Ness:2015} \\
        \vdots & & all columns from \apogee\ \DR\ \texttt{allStar} file \\
        \hline
        \multicolumn{3}{c}{\textit{(\nunimodal\ rows)}}
    \end{tabular}
    \caption{Description of data table containing summary information for all
    stars in the high-$K$, unimodal sample:
    Stars that likely have companions and have well-determined orbital
    parameters.
    All orbital parameter values are from the maximum \textit{a posteriori}
    (MAP) posterior sample (either from \thejoker, or from \package{emcee}).
    All columns ending in \texttt{\_err} are estimates of the standard-deviation
    of the posterior samples, $\sigma$, computed using the median absolute
    deviation, $\textrm{MAD}$, as $\sigma \approx 1.5 \times \textrm{MAD}$.
    }
    \label{tbl:highK-unimodal}
\end{table}
