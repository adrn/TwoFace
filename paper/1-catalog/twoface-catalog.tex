\documentclass[modern, letterpaper]{aastex61}

% to-do list
% ----------
% - write a first draft of the introduction
% - list our assumptions for detection & characterization (TheJoker)
% - Check abstract numbers (~80,000 RG stars with 3 or more good visits?)

% style notes
% -----------
% - This file generates by Makefile; don't be typing ``pdflatex'' or some BS.
% - Line break between sentences to make the git diffs readable.
% - Use \, as a multiply operator.
% - Reserve () for function arguments; use [] or {} for outer shit.
% - Always prior pdf or posterior pdf, never prior or posterior (that's your
%   arse).
% - Use \sectionname not Section, \figname not Figure, \documentname not Article
%   \tablename not Table, \eqname not Equation.
% - Make sure that two assumptions in the two assumption lists only have the
%   same name if they really are the same assumption: These are proper names!
% - Hyphenate binary-star when it is an adjective, not when it is a noun!
% - Where there are defined math symbols (like \pars), use them!

% other notes
% -----------
% - Binary engulfment reference: https://arxiv.org/pdf/1002.2216.pdf

\include{gitstuff}
% Load common packages
% \usepackage{microtype}  % ALWAYS!
\usepackage{amsmath}
\usepackage{amsfonts}
\usepackage{amssymb}
\usepackage{booktabs}

\usepackage{graphicx}
\usepackage{color}

\definecolor{cbblue}{HTML}{3182bd}
\usepackage{hyperref}
\definecolor{linkcolor}{rgb}{0.02,0.35,0.55}
\definecolor{citecolor}{rgb}{0.45,0.45,0.45}
\hypersetup{colorlinks=true,linkcolor=linkcolor,citecolor=citecolor,
            filecolor=linkcolor,urlcolor=linkcolor}
\hypersetup{pageanchor=true}

\newcommand{\documentname}{\textsl{Article}}
\newcommand{\sectionname}{Section}
\renewcommand{\figurename}{Figure}
\newcommand{\eqname}{Equation}
\renewcommand{\tablename}{Table}

% Packages / projects / programming
\newcommand{\package}[1]{\textsl{#1}}
\newcommand{\acronym}[1]{{\small{#1}}}
\newcommand{\github}{\package{GitHub}}
\newcommand{\python}{\package{Python}}
\newcommand{\emcee}{\project{emcee}}

% Missions
\newcommand{\project}[1]{\textsl{#1}}

% For referee
\newcommand{\changes}[1]{{\color{red} #1}}

% Stats / probability
\newcommand{\given}{\,|\,}
\newcommand{\norm}{\mathcal{N}}

% Maths
\newcommand{\dd}{\mathrm{d}}
\newcommand{\transpose}[1]{{#1}^{\mathsf{T}}}
\newcommand{\inverse}[1]{{#1}^{-1}}
\newcommand{\argmin}{\operatornamewithlimits{argmin}}
\newcommand{\mean}[1]{\left< #1 \right>}

% Unit shortcuts
\newcommand{\msun}{\ensuremath{\mathrm{M}_\odot}}
\newcommand{\kms}{\ensuremath{\mathrm{km}~\mathrm{s}^{-1}}}
\newcommand{\pc}{\ensuremath{\mathrm{pc}}}
\newcommand{\kpc}{\ensuremath{\mathrm{kpc}}}
\newcommand{\kmskpc}{\ensuremath{\mathrm{km}~\mathrm{s}^{-1}~\mathrm{kpc}^{-1}}}

% Misc.
\newcommand{\bs}[1]{\boldsymbol{#1}}
\definecolor{mahogany}{RGB}{165,15,21}
\newcommand{\resp}[1]{{\color{mahogany}#1}}

% Astronomy
\newcommand{\DM}{{\rm DM}}
\newcommand{\feh}{\ensuremath{{[{\rm Fe}/{\rm H}]}}}
\newcommand{\df}{\acronym{DF}}

% TO DO
\newcommand{\todo}[1]{{\color{red} TODO: #1}}


% adjust AAS-TEX shit
% \setlength{\parindent}{1.1\baselineskip}

% define macros for text
\newcommand{\apogee}{\project{\acronym{APOGEE}}}
\newcommand{\sdssiii}{\project{\acronym{SDSS-III}}}
\newcommand{\thejoker}{\project{The~Joker}}
\newcommand{\thecannon}{\project{The~Cannon}}
\newcommand{\DR}{\acronym{DR14}}
\newcommand{\RC}{\acronym{RC}}
\newcommand{\RGB}{\acronym{RGB}}

\newcommand{\nRC}{XX}
\newcommand{\nRGB}{XX}
\newcommand{\ntotal}{XX}
\newcommand{\ncompanions}{XX}

% define macros for math
\newcommand{\hyperpars}{\gamma}
\newcommand{\pars}{\theta}

% for response to referee
% \renewcommand{\resp}[1]{#1}

\shortauthors{Price-Whelan et al.}

\begin{document}\sloppy\sloppypar\raggedbottom\frenchspacing % trust me

\title{Binary companions of red giant stars with APOGEE I: \\
       catalog and companions that survive the common envelope}

\author[0000-0003-0872-7098]{Adrian~M.~Price-Whelan}
\affiliation{Department of Astrophysical Sciences,
             Princeton University, Princeton, NJ 08544, USA}
\email{adrn@astro.princeton.edu}
\correspondingauthor{Adrian M. Price-Whelan}

\author[0000-0003-2866-9403]{David~W.~Hogg}
\affiliation{Max-Planck-Institut f\"ur Astronomie,
             K\"onigstuhl 17, D-69117 Heidelberg, Germany}
\affiliation{Center for Cosmology and Particle Physics,
             Department of Physics,
             New York University, 726 Broadway,
             New York, NY 10003, USA}
\affiliation{Center for Data Science,
             New York University, 60 Fifth Ave,
             New York, NY 10011, USA}
\affiliation{Flatiron Institute,
             Simons Foundation,
             162 Fifth Avenue,
             New York, NY 10010, USA}

\author{Hans-Walter~Rix}
\affiliation{Max-Planck-Institut f\"ur Astronomie,
             K\"onigstuhl 17, D-69117 Heidelberg, Germany}

% \author{Jason~Cao}
% \affiliation{Center for Cosmology and Particle Physics,
%              Department of Physics,
%              New York University, 726 Broadway,
%              New York, NY 10003, USA}

\begin{abstract}\noindent % trust me
% Context
Repeat radial-velocity measurements of stars can be used to identify stellar-,
sub-stellar, and planetary-mass companions.
Even few observations are useful for detecting companions, though such data are
difficult to use for constraining the physical properties of companions because
permitted orbital solutions are highly degenerate.
% Aims
Here we perform a search for secondary companions of red giant branch (\RGB)
stars using data from the \apogee\ survey (\DR), which has measured multi-epoch
($\geq 3$) but sparse radial velocities for $> 80,000$ \RGB\ stars.
In some cases, where the primary-star masses are known, the mass-function
degeneracy is broken and the secondary-companion masses ($m_2\,\sin i$) can be
inferred.
% Methods
We use a custom-built Monte Carlo sampler (\thejoker) combined with stellar
parameters and primary mass estimates to deliver (often highly multi-modal)
posterior beliefs about companion mass, pericenter distance, and other orbital
parameters for the stars.
We compare the companion populations of red clump (\RC) versus \RGB\ stars at
similar surface gravity and temperature, and study the metallicity dependence
of companion properties.
% Results
We provide a catalog of \ncompanions\ companions for which the companion
properties can be confidently determined, along with posterior samplings for all
\ntotal\ stars in the parent sample.
\todo{We find differences between the companions around \RC\ and \RGB\
stars that are in/consistent with the theoretical prediction that
stellar companions should not survive the common-envelope \RGB\ phase.}
\todo{We additionally find trends in XX with chemical abundances.}

\end{abstract}

\keywords{
  binaries:~spectroscopic
  ---
  methods:~data~analysis
  ---
  methods:~statistical
  ---
  planets~and~satellites:~fundamental~parameters
  ---
  surveys
  ---
  techniques:~radial~velocities
}

\section{Introduction} \label{sec:intro}

\todo{An alternate approach to this intro is to start with an overview of
      binary stars, radial velocity measurements of binarity in general...}

\todo{As is, the intro is missing many citations}

Time-domain radial-velocity measurements of stars contain information about
massive companions: even with two successive observations of a single star, a
difference in the measured radial velocities implies the existence of at least
one companion.
However, with few or imprecise radial-velocity measurements, the orbital
properties of the companion(s) are very poorly constrained.
Most prior searches for companions using survey RV data have therefore
restricted their searches to only sources with many epochs, so that the orbital
solution can be unambiguously determined.
The vast majority of spectroscopic targets with repeat observations in the
largest (by number of objects) stellar spectroscopic surveys are in the opposite
regime: targets are often observed just a few times with sparse, non-uniform
phase coverage.

% With such data, even if the period or other generic orbital parameters
% are very poorly constrained, limits on companion mass or pericentric distance
% of the companion may still XX.

If there are only a few radial-velocity measurements made per star, and the
companion spectrum is not observed, any measured radial-velocity values will be
consistent with many different combinations of (primary) orbital parameters
(period, amplitude, eccentricity, etc.).
To identify companions to the typical star observed in a spectroscopic survey,
we therefore face at least one major challenge: how, given a small number of
observations of a primary star, do we reliably obtain posterior information
about the binary-system properties?
In general, the likelihood function---and the posterior probability distribution
function (pdf) under any reasonable prior pdf---will be highly multimodal, and
many of the modes will have comparable integrated probability density.
For example, with just two radial-velocity measurements, a harmonic series of
period modes will exist in the likelihood function.

We have solved this problem previously, though with limitations (to be discussed
more below), with \thejoker\ (\citealt{Price-Whelan:2017}).
\thejoker\ is a Monte Carlo rejection sampler that is computationally expensive
but probabilistically righteous:
It delivers independent posterior pdf samples for single-companion binary
orbital parameters, given any number of radial-velocity measurements.
Here we use \thejoker\ to generate posterior pdf samples for \todo{which?} stars
observed by the \apogee\ surveys (see \sectionname~\ref{sec:data};
\citealt{Majewski:2015}).

The \apogee\ surveys primarily target red-giant-branch (\RGB) stars, which are
ideal for the study of single-line binary systems.
For one, because they are so luminous, they are unlikely (in general) to have
equally-bright companions, and their spectra are therefore well-approximated or
fit as single-line objects.
The subset of \RGB\ stars in the ``red clump'' (\RC) are even more powerful as
they are standard candles and have masses that can be estimated using
spectroscopy (using dredged-up elements; \citealt{Martig:2016,Ness:2016}).
With primary-star mass estimates, the binary-orbit fitting will return
$m_2\,\sin i$ estimates for the secondary, and not just estimates of the
mass-function.
Additionally, the \apogee\ pipelines (\citealt{Garcia-Perez:2016}) and also
\thecannon\ (\citealt{Ness:2015}) produce detailed abundance estimates for \RGB\
and \RC\ stars.
If there are causal relationships between chemical abundances and binary
companions---as are expected---these should be measurable here.

By making cuts on this library of posterior pdf samples (described in detail in
\sectionname~\ref{sec:whatever}), we deliver a catalog of binary-star systems
from the \apogee\ survey with no cuts or constraints on data quality or volume.
We additionally highlight the subset of this catalog with \RC\ star primaries,
for which we can compute companion mass and pericentric distance.

\section{Data} \label{sec:data}

\todo{HOGG: fill in this ish}

HOGG: Why and how does \apogee\ rock it?

ASPCAP: \citealt{Garcia-Perez:2016}
SDSSIV: \citealt{Blanton:2017}

Overview of what \apogee\ is, and what data we are using.

How we updated the individual-visit radial velocity measurements.

What we did (if anything) with the missing visits.

\section{Methods}

\subsection{Orbit fitting and velocity modeling}\label{sec:fitting}

For every source in the sample of \apogee\ stars defined in
\sectionname~\ref{sec:data}, we obtain a posterior sampling in binary-system
parameter space, treating it as a single-lined (SB1) spectroscopic binary system
with a single companion.
This sampling is performed under a relatively uninformative prior pdf, and the
resulting posterior samplings are used to discover and characterize individual
binary-star systems and generate a catalog of companions.
We use \thejoker\ to perform the posterior samplings; we briefly describe the
algorithm below, but a full description is given in previous work
(\citealt{Price-Whelan:2017}).
In this \sectionname, we describe the assumptions and method used to generate
individual-system samplings.

We perform our individual system fits---that is, make our posterior
samplings---under the following assumptions:
\begin{description}
\item[no multiplets] All radial-velocity variations of the primary star are
  induced by a single companion.
  This is motivated by the idea that triple star systems are usually
  hierarchical so that the period of the inner binary is typically much shorter
  than the orbital period of the outer body (\todo{CITATION NEEDED}).
  At present, we ignore the possibility of higher-order multiple systems.
  \todo{revisit this - do we want to also sample over a long-term trend?}
\item[Kepler] Related to the first point, all velocity variations of the primary
  are gravitational, and we therefore ignore the possibility of intrinsic
  variation from, e.g., stellar oscillations.
\item[SB1] All spectra are single-lined; that is, we assume that the secondary
  is significantly fainter and is thus undetected in the spectra.
  This assumption is motivated by the fact that we expect \RGB\ stars to be
  substantially more luminous than their typical companion.
  However, there are known double-lined binary stars in the \apogee\ catalog
  (El-Badry et al., in prep.), and an unknown fraction of \RGB--\RGB\ binaries.
\item[simple noise model] Measurements are unbiased and noise estimates are
  correct up to an \todo{unknown} extra variance (the ``jitter'').
  All noise contributions result in Gaussian uncertainties on the individual
  radial-velocity measurements.
\end{description}

\thejoker\ is a custom-built Monte Carlo sampler designed to produce independent
posterior samples in orbital parameters, given radial-velocity measurements
under the assumptions above.
Our parametrization is similar to the notation in \citet{Murray:2010}:
The radial velocity $v$ at time $t$ is given by
\begin{equation}
  v(t;\bs{\theta}) = v_0 + K\,[\cos\left(\omega + f(t; e, P, \phi_0)\right) +
    e\,\cos\omega]
\end{equation}
where $\bs{\theta} = (P,e,\phi_0,\omega,K,v_0)$---period, eccentricity, initial
phase, argument of pericenter, velocity semi-amplitude, Barycentric velocity---
and the true anomaly, $f$, is a function of the specified parameters (see
\sectionname~2 of \citealt{Price-Whelan:2017} or \eqname~63 in
\citealt{Murray:2010}).

\thejoker\ was designed for the extremely multi-modal pdfs expected when the
number of radial-velocity measurements of a source is small, or the data have
low signal-to-noise.
While other Markov Chain Monte Carlo (MCMC) methods have difficulty producing
independent samples with such data, \thejoker\ succeeds by brute force.
After generating an initial (very large) library of prior samples from an
assumed prior pdf (see below), the (typically multi-modal) likelihood is
evaluated at each sample and used to rejection sample.
In practice, given a number of requested samples for each star, the sampling
proceeds iteratively: since it is easier to accept samples when the data is
sparse or noisy, far more prior sample draws (and thus likelihood
evaluations) must occur under very constraining data.

In addition to generating posterior samples for each source, we also compute the
fully marginalized likelihood (FML) under two other models: (1) a model in which
the radial velocity of the source is constant, and (2) a model in which the
radial-velocity variations are purely linear.
We later use these FML values to define cuts on our full catalog of posterior
samples to produce a catalog of companions.

\todo{equations?}

Examples of outputs and etc.

\section{Results}

\subsection{Catalog / posterior samples whatever} \label{sec:full-catalog}

\todo{APW}

\subsection{A catalog of confident companions} \label{sec:conf-companions}

\todo{APW}

Thresholding on what now?

Catalog.

Some highlights from this catalog.

\subsection{Differences in companions of RC and RGB stars}
\label{sec:rc-rgb}

\todo{APW}

\subsection{Trends in RGB companions with chemical abundances}
\label{sec:rgb-chemistry}

\todo{APW}

\section{Discussion}

Return to our various assumptions and make sure that we discuss them
\emph{by name} here.

\acknowledgements

It is a pleasure to thank XX.

The authors are pleased to acknowledge that the work reported on in this
paper was substantially performed at the TIGRESS high performance computer
center at Princeton University which is jointly supported by the Princeton
Institute for Computational Science and Engineering and the Princeton
University Office of Information Technology's Research Computing department.

\software{
The code used in this project is available from
\url{https://github.com/adrn/TwoFace} under the MIT open-source
software license.
This research utilized the following open-source \python\ packages:
    \package{Astropy} (\citealt{Astropy-Collaboration:2013}),
    % \package{corner} (\citealt{Foreman-Mackey:2016}),
    % \package{emcee} (\citealt{Foreman-Mackey:2013ascl}),
    \package{IPython} (\citealt{Perez:2007}),
    \package{matplotlib} (\citealt{Hunter:2007}),
    \package{numpy} (\citealt{Van-der-Walt:2011}),
    \package{scipy} (\url{https://www.scipy.org/}),
    \package{sqlalchemy} (\url{https://www.sqlalchemy.org/}).
}

\facility{\sdssiii, \apogee}

\bibliographystyle{aasjournal}
\bibliography{refs}

\end{document}
