\documentclass[modern, letterpaper]{aastex61}

% to-do list
% ----------
% - write a first draft of the introduction
% - list our assumptions for detection & characterization (TheJoker)

% style notes
% -----------
% - This file generates by Makefile; don't be typing ``pdflatex'' or some BS.
% - Line break between sentences to make the git diffs readable.
% - Use \, as a multiply operator.
% - Reserve () for function arguments; use [] or {} for outer shit.
% - Always prior pdf or posterior pdf, never prior or posterior (that's your
%   arse).
% - Use \sectionname not Section, \figname not Figure, \documentname not Article
%   \tablename not Table, \eqname not Equation.
% - Make sure that two assumptions in the two assumption lists only have the
%   same name if they really are the same assumption: These are proper names!
% - Hyphenate binary-star when it is an adjective, not when it is a noun!
% - Where there are defined math symbols (like \pars), use them!

% other notes
% -----------
% - Binary engulfment reference: https://arxiv.org/pdf/1002.2216.pdf

\include{gitstuff}
% Load common packages
% \usepackage{microtype}  % ALWAYS!
\usepackage{amsmath}
\usepackage{amsfonts}
\usepackage{amssymb}
\usepackage{booktabs}

\usepackage{graphicx}
\usepackage{color}

\definecolor{cbblue}{HTML}{3182bd}
\usepackage{hyperref}
\definecolor{linkcolor}{rgb}{0.02,0.35,0.55}
\definecolor{citecolor}{rgb}{0.45,0.45,0.45}
\hypersetup{colorlinks=true,linkcolor=linkcolor,citecolor=citecolor,
            filecolor=linkcolor,urlcolor=linkcolor}
\hypersetup{pageanchor=true}

\newcommand{\documentname}{\textsl{Article}}
\newcommand{\sectionname}{Section}
\renewcommand{\figurename}{Figure}
\newcommand{\eqname}{Equation}
\renewcommand{\tablename}{Table}

% Packages / projects / programming
\newcommand{\package}[1]{\textsl{#1}}
\newcommand{\acronym}[1]{{\small{#1}}}
\newcommand{\github}{\package{GitHub}}
\newcommand{\python}{\package{Python}}
\newcommand{\emcee}{\project{emcee}}

% Missions
\newcommand{\project}[1]{\textsl{#1}}

% For referee
\newcommand{\changes}[1]{{\color{red} #1}}

% Stats / probability
\newcommand{\given}{\,|\,}
\newcommand{\norm}{\mathcal{N}}

% Maths
\newcommand{\dd}{\mathrm{d}}
\newcommand{\transpose}[1]{{#1}^{\mathsf{T}}}
\newcommand{\inverse}[1]{{#1}^{-1}}
\newcommand{\argmin}{\operatornamewithlimits{argmin}}
\newcommand{\mean}[1]{\left< #1 \right>}

% Unit shortcuts
\newcommand{\msun}{\ensuremath{\mathrm{M}_\odot}}
\newcommand{\kms}{\ensuremath{\mathrm{km}~\mathrm{s}^{-1}}}
\newcommand{\mps}{\ensuremath{\mathrm{m}~\mathrm{s}^{-1}}}
\newcommand{\pc}{\ensuremath{\mathrm{pc}}}
\newcommand{\kpc}{\ensuremath{\mathrm{kpc}}}
\newcommand{\kmskpc}{\ensuremath{\mathrm{km}~\mathrm{s}^{-1}~\mathrm{kpc}^{-1}}}

% Misc.
\newcommand{\bs}[1]{\boldsymbol{#1}}
\definecolor{mahogany}{RGB}{165,15,21}
\newcommand{\resp}[1]{{\color{mahogany}#1}}

% Astronomy
\newcommand{\DM}{{\rm DM}}
\newcommand{\feh}{\ensuremath{{[{\rm Fe}/{\rm H}]}}}
\newcommand{\df}{\acronym{DF}}

% TO DO
\newcommand{\todo}[1]{{\color{red} TODO: #1}}


% adjust AAS-TEX shit
\setlength{\parindent}{1.1\baselineskip}

% define macros for text
\newcommand{\apogee}{\project{\acronym{APOGEE}}}
\newcommand{\sdssiii}{\project{\acronym{SDSS-III}}}
\newcommand{\thejoker}{\project{The~Joker}}
\newcommand{\thecannon}{\project{The~Cannon}}
\newcommand{\DR}{\acronym{DR13}}
\newcommand{\RC}{\acronym{RC}}
\newcommand{\RGB}{\acronym{RGB}}

\newcommand{\nRC}{XX}
\newcommand{\nRGB}{XX}
\newcommand{\ntotal}{XX}
\newcommand{\ncompanions}{XX}

% define macros for math
\newcommand{\hyperpars}{\gamma}
\newcommand{\pars}{\theta}

% for response to referee
% \renewcommand{\resp}[1]{#1}

\shortauthors{Price-Whelan et al.}

\begin{document}\sloppy\sloppypar\raggedbottom\frenchspacing % trust me

\title{Binary companions of red giant stars I: \\
       catalog and companions that survive the common envelope}

\author[0000-0003-0872-7098]{Adrian~M.~Price-Whelan}
\affiliation{Department of Astrophysical Sciences,
             Princeton University, Princeton, NJ 08544, USA}
\email{adrn@astro.princeton.edu}
\correspondingauthor{Adrian M. Price-Whelan}

\author[0000-0003-2866-9403]{David~W.~Hogg}
\affiliation{Max-Planck-Institut f\"ur Astronomie,
             K\"onigstuhl 17, D-69117 Heidelberg, Germany}
\affiliation{Center for Cosmology and Particle Physics,
             Department of Physics,
             New York University, 726 Broadway,
             New York, NY 10003, USA}
\affiliation{Center for Data Science,
             New York University, 60 Fifth Ave,
             New York, NY 10011, USA}
\affiliation{Flatiron Institute,
             Simons Foundation,
             162 Fifth Avenue,
             New York, NY 10010, USA}

\author{Hans-Walter~Rix}
\affiliation{Max-Planck-Institut f\"ur Astronomie,
             K\"onigstuhl 17, D-69117 Heidelberg, Germany}

\author{Jason~Cao}
\affiliation{Center for Cosmology and Particle Physics,
             Department of Physics,
             New York University, 726 Broadway,
             New York, NY 10003, USA}

\begin{abstract}\noindent % trust me
% Context
Repeat radial-velocity measurements of stars can be used to find stellar,
sub-stellar, and planetary companions.
The \apogee\ survey (\DR) has measured multi-epoch but sparse radial velocities
for $> XX$ (mainly red giant) stars.
Though even few observations are useful for detecting companions, such data are
difficult to use for constraining the physical properties of companions because
permitted orbital solutions are highly degenerate.
% Aims
Here we perform a search for secondary companions of \nRC\ red clump (\RC) and
\nRGB\ red giant branch (\RGB) stars that have similar radii to \RC\ stars
using data from the \apogee\ survey.
Because the primary-star masses are known, the mass-function degeneracy is
broken and the secondary-companion masses ($m\,\sin i$) can be inferred.
% Methods
We use a custom-built Monte Carlo sampler (\thejoker) combined with prior
measurements of primary masses and stellar parameters to deliver (often highly
multi-modal) posterior beliefs about companion mass, pericenter distance, and
other orbital parameters for the \RC\ and \RGB\ stars.
% Results
We deliver a catalog of \ncompanions\ companions \todo{that meet some criteria},
and posterior samplings for all \ntotal\ stars in the parent sample.
We find \todo{describe} differences between the companions around \RC\ and \RGB\
stars that are \todo{in/consistent} with the theoretical prediction that
stellar companions should not survive the common-envelope \RGB\ phase.
We additionally find trends with chemical abundances, and we find tentative
evidence for ZZZ and WWW.
\end{abstract}

\keywords{
  binaries:~spectroscopic
  ---
  methods:~data~analysis
  ---
  methods:~statistical
  ---
  planets~and~satellites:~fundamental~parameters
  ---
  surveys
  ---
  techniques:~radial~velocities
}

\section{Introduction} \label{sec:intro}

Time-domain radial-velocity measurements of stars contain information about
massive companions: even with two successive observations of a single star, a
difference in the measured radial velocities implies the existence of at least
one companion.
However, with few or imprecise radial-velocity measurements, the orbital
properties of the companion(s) are very poorly constrained.
Most prior searches for companions using survey RV data have therefore
restricted their searches to only sources with many epochs, so that the orbital
solution can be unambiguously determined.
The vast majority of spectroscopic targets with repeat observations in the
largest (by number of objects) stellar spectroscopic surveys are in the opposite
regime: targets are often observed just a few times with sparse, non-uniform
phase coverage.

% With such data, even if the period or other generic orbital parameters
% are very poorly constrained, limits on companion mass or pericentric distance
% of the companion may still XX.

If there are only a few radial-velocity measurements made per star, and the
companion spectrum is not observed, any measured radial-velocity values will be
consistent with many different combinations of (primary) orbital parameters
(period, amplitude, eccentricity, etc.).
To identify companions to the typical star observed in a spectroscopic survey,
we therefore face at least one major challenge: how, given a small number of
observations of a primary star, do we reliably obtain posterior information
about the binary-system properties?
In general, the likelihood function---and the posterior probability distribution
function (pdf) under any reasonable prior pdf---will be highly multimodal, and
many of the modes will have comparable integrated probability density.
For example, with just two radial-velocity measurements, a harmonic series of
period modes will exist in the likelihood function.

We have solved this problem previously, though with limitations (to be discussed
more below), with \thejoker\ (\citealt{Price-Whelan:2017}).
\thejoker\ is a Monte Carlo rejection sampler that is computationally expensive
but probabilistically righteous:
It delivers independent posterior pdf samples for single-companion binary
orbital parameters, given any number of radial-velocity measurements.
Here we use \thejoker\ to generate posterior pdf samples for \todo{which?} stars
observed by the \apogee\ surveys (see \sectionname~\ref{sec:data};
\citealt{Majewski:2015}).

The \apogee\ surveys primarily target red-giant-branch (\RGB) stars, which are
ideal for the study of single-line binary systems.
For one, because they are so luminous, they are unlikely (in general) to have
equally-bright companions, and their spectra are therefore well-approximated or
fit as single-line objects.
The subset of \RGB\ stars in the ``red clump'' (\RC) are even more powerful as
they are standard candles and have masses that can be estimated using
spectroscopy (using dredged-up elements; \citealt{Martig:2016,Ness:2016}).
With primary-star mass estimates, the binary-orbit fitting will return
$m_2\,\sin i$ estimates for the secondary, and not just estimates of the
mass-function.
Additionally, the \apogee\ pipelines (\citealt{Garcia-Perez:2016}) and also
\thecannon\ (\citealt{Ness:2015}) produce detailed abundance estimates for \RGB\
and \RC\ stars.
If there are causal relationships between chemical abundances and binary
companions---as are expected---these should be measurable here.

By making cuts on this library of posterior pdf samples (described in detail in
\sectionname~\ref{sec:whatever}), we deliver a catalog of binary-star systems
from the \apogee\ survey with no cuts or constraints on data quality or volume.
We additionally highlight the subset of this catalog with \RC\ star primaries,
for which we can compute companion mass and pericentric distance.

\section{Data} \label{sec:data}

\todo{HOGG: fill in this ish}

HOGG: Why and how does \apogee\ rock it?

ASPCAP: \citealt{Garcia-Perez:2016}
SDSSIV: \citealt{Blanton:2017}

Overview of what \apogee\ is, and what data we are using.

How we updated the individual-visit radial velocity measurements.

What we did (if anything) with the missing visits.

\section{Methods}

\subsection{Re-measured radial velocities}\label{sec:rv-remeasure}

\todo{HOGG: fill in this ish}

\subsection{Detection and orbit fitting}\label{sec:fitting}

Our approach here is to proceed in two phases.
In the first phase, we obtain a posterior sampling in binary-system
parameter space for every individual star, treating it as a
single-lined (SB1) spectroscopic binary system with a single
companion.
This sampling is performed under a relatively uninformative prior pdf,
which will be called the ``\emph{interim prior}''.
These samplings are used to discover and characterize individual binary-star
systems, and generate (among other things) a catalog of binaries.
In the second phase, we perform a hierarchical inference of the binary
population, using (responsibly) these samplings under the interim
prior as inputs.
In the second phase we use all samplings of all stars, not just those that
pass our thresholds for discovery as binary systems.
In this \sectionname, we describe the first phase, that is, the
individual-system sampling phase for discovery and characterization.

We perform our individual system fits---that is, make our posterior
samplings---under the following assumptions:
\begin{description}
\item[no trinaries] whatevs: No trinaries or beyond. Obviously wrong!
\item[SB1] whatevs: There is no significant light contributed by the secondary.
\item[Kepler] whatevs: No non-gravitational contributions to RV history.
\item[noise model] whatevs: Measurements are unbiased and noise estimates are correct up to jitter; everything Gaussian.
\item[interim prior] We adopt a prior pdf on binary-system parameters
  with the following properties...
\end{description}

Summary of \thejoker\ and what it does; that is, that it obeys the
relevant parts of our assumptions.

Any modifications or knob-settings for \thejoker?

Examples of outputs and etc.

\section{Results}

\subsection{Catalog / posterior samples whatever} \label{sec:full-catalog}

\todo{APW}

\subsection{A catalog of red-clump binary systems} \label{sec:rc-catalog}

\todo{APW}

Thresholding on what now?

Catalog.

Some highlights from this catalog.

\subsection{Differences in companions of RC and RGB stars} \label{sec:rc-rgb}

\todo{APW}

\section{Discussion}

Return to our various assumptions and make sure that we discuss them
\emph{by name} here.

\acknowledgements

It is a pleasure to thank XX.

The authors are pleased to acknowledge that the work reported on in this
paper was substantially performed at the TIGRESS high performance computer
center at Princeton University which is jointly supported by the Princeton
Institute for Computational Science and Engineering and the Princeton
University Office of Information Technology's Research Computing department.

\software{
The code used in this project is available from
\url{https://github.com/adrn/TwoFace} under the MIT open-source
software license.
This research utilized the following open-source \python\ packages:
    \package{Astropy} (\citealt{Astropy-Collaboration:2013}),
    % \package{corner} (\citealt{Foreman-Mackey:2016}),
    % \package{emcee} (\citealt{Foreman-Mackey:2013ascl}),
    \package{IPython} (\citealt{Perez:2007}),
    \package{matplotlib} (\citealt{Hunter:2007}),
    \package{numpy} (\citealt{Van-der-Walt:2011}),
    \package{scipy} (\url{https://www.scipy.org/}),
    \package{sqlalchemy} (\url{https://www.sqlalchemy.org/}).
}

\facility{\sdssiii, \apogee}

\bibliographystyle{aasjournal}
\bibliography{refs}

\end{document}
