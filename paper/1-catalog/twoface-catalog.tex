\documentclass[modern, letterpaper]{aastex62}

% to-do list
% ----------
% - write a first draft of the introduction
% - list our assumptions for detection & characterization (TheJoker)
% - Check abstract numbers (~80,000 RG stars with 3 or more good visits?)

% style notes
% -----------
% - This file generates by Makefile; don't be typing ``pdflatex'' or some BS.
% - Line break between sentences to make the git diffs readable.
% - Use \, as a multiply operator.
% - Reserve () for function arguments; use [] or {} for outer shit.
% - Always prior pdf or posterior pdf, never prior or posterior (that's your
%   arse).
% - Use \sectionname not Section, \figname not Figure, \documentname not Article
%   \tablename not Table, \eqname not Equation.
% - Make sure that two assumptions in the two assumption lists only have the
%   same name if they really are the same assumption: These are proper names!
% - Hyphenate binary-star when it is an adjective, not when it is a noun!
% - Where there are defined math symbols (like \pars), use them!

% other notes
% -----------
% - Binary engulfment reference: https://arxiv.org/pdf/1002.2216.pdf

\include{gitstuff}
% Load common packages
% \usepackage{microtype}  % ALWAYS!
\usepackage{amsmath}
\usepackage{amsfonts}
\usepackage{amssymb}
\usepackage{booktabs}

\usepackage{graphicx}
\usepackage{color}

\definecolor{cbblue}{HTML}{3182bd}
\usepackage{hyperref}
\definecolor{linkcolor}{rgb}{0.02,0.35,0.55}
\definecolor{citecolor}{rgb}{0.45,0.45,0.45}
\hypersetup{colorlinks=true,linkcolor=linkcolor,citecolor=citecolor,
            filecolor=linkcolor,urlcolor=linkcolor}
\hypersetup{pageanchor=true}

\newcommand{\documentname}{\textsl{Article}}
\newcommand{\sectionname}{Section}
\renewcommand{\figurename}{Figure}
\newcommand{\eqname}{Equation}
\renewcommand{\tablename}{Table}

% Packages / projects / programming
\newcommand{\package}[1]{\textsl{#1}}
\newcommand{\acronym}[1]{{\small{#1}}}
\newcommand{\github}{\package{GitHub}}
\newcommand{\python}{\package{Python}}
\newcommand{\emcee}{\project{emcee}}

% Missions
\newcommand{\project}[1]{\textsl{#1}}

% For referee
\newcommand{\changes}[1]{{\color{red} #1}}

% Stats / probability
\newcommand{\given}{\,|\,}
\newcommand{\norm}{\mathcal{N}}

% Maths
\newcommand{\dd}{\mathrm{d}}
\newcommand{\transpose}[1]{{#1}^{\mathsf{T}}}
\newcommand{\inverse}[1]{{#1}^{-1}}
\newcommand{\argmin}{\operatornamewithlimits{argmin}}
\newcommand{\mean}[1]{\left< #1 \right>}

% Unit shortcuts
\newcommand{\msun}{\ensuremath{\mathrm{M}_\odot}}
\newcommand{\kms}{\ensuremath{\mathrm{km}~\mathrm{s}^{-1}}}
\newcommand{\mps}{\ensuremath{\mathrm{m}~\mathrm{s}^{-1}}}
\newcommand{\pc}{\ensuremath{\mathrm{pc}}}
\newcommand{\kpc}{\ensuremath{\mathrm{kpc}}}
\newcommand{\kmskpc}{\ensuremath{\mathrm{km}~\mathrm{s}^{-1}~\mathrm{kpc}^{-1}}}

% Misc.
\newcommand{\bs}[1]{\boldsymbol{#1}}
\definecolor{mahogany}{RGB}{165,15,21}
\newcommand{\resp}[1]{{\color{mahogany}#1}}

% Astronomy
\newcommand{\DM}{{\rm DM}}
\newcommand{\feh}{\ensuremath{{[{\rm Fe}/{\rm H}]}}}
\newcommand{\df}{\acronym{DF}}

% TO DO
\newcommand{\todo}[1]{{\color{red} TODO: #1}}


% adjust AAS-TEX shit
% \setlength{\parindent}{1.1\baselineskip}

\graphicspath{{figures/}}

% define macros for text
\newcommand{\apogee}{\project{\acronym{APOGEE}}}
\newcommand{\sdssiv}{\project{\acronym{SDSS-IV}}}
\newcommand{\thejoker}{\project{The~Joker}}
\newcommand{\thecannon}{\project{The~Cannon}}
\newcommand{\DR}{\acronym{DR14}}
\newcommand{\RC}{\acronym{RC}}
\newcommand{\RGB}{\acronym{RGB}}

\newcommand{\nprior}{536,870,912}
\newcommand{\nposterior}{256}
\newcommand{\nstars}{96,231}
\newcommand{\nvisits}{397,559}
\newcommand{\ncontrol}{16,384}
\newcommand{\nhighK}{4,510}
\newcommand{\nunimodal}{XX}
\newcommand{\nmultimodal}{XX}

% for response to referee
% \renewcommand{\resp}[1]{#1}

\shortauthors{Price-Whelan et al.}

\begin{document}\sloppy\sloppypar\raggedbottom\frenchspacing % trust me

\title{Binary companions of red giant stars in \apogee\ \DR}

\author[0000-0003-0872-7098]{Adrian~M.~Price-Whelan}
\affiliation{Department of Astrophysical Sciences,
             Princeton University, Princeton, NJ 08544, USA}
\email{adrn@astro.princeton.edu}
\correspondingauthor{Adrian M. Price-Whelan}

\author[0000-0003-2866-9403]{David~W.~Hogg}
\affiliation{Max-Planck-Institut f\"ur Astronomie,
             K\"onigstuhl 17, D-69117 Heidelberg, Germany}
\affiliation{Center for Cosmology and Particle Physics,
             Department of Physics,
             New York University, 726 Broadway,
             New York, NY 10003, USA}
\affiliation{Center for Data Science,
             New York University, 60 Fifth Ave,
             New York, NY 10011, USA}
\affiliation{Flatiron Institute,
             Simons Foundation,
             162 Fifth Avenue,
             New York, NY 10010, USA}

\author[0000-0003-4996-9069]{Hans-Walter~Rix}
\affiliation{Max-Planck-Institut f\"ur Astronomie,
             K\"onigstuhl 17, D-69117 Heidelberg, Germany}

\author{APOGEE Team}

\begin{abstract}\noindent % trust me
% Context
Repeat radial-velocity measurements of stars can be used to identify stellar,
sub-stellar, and planetary-mass companions.
Even a very small number of observation epochs can be useful for detecting
companions, though such data can be difficult to use for characterizing
individual systems:
There can be multiple qualitatively different orbital solutions that fit the
data.
For these reasons, we custom-built a Monte Carlo sampler (\thejoker) that
delivers reliable (and often highly multi-modal) posterior samplings for
companion orbital parameters given sparse radial-velocity data.
% Aims
Here we perform a search for secondary companions of \nstars\ red-giant stars
observed in the \apogee\ survey (\DR) with $\geq 3$ spectroscopic epochs.
% Methods
We select stars with probable companions by making a cut on our posterior belief
about the amplitude of the stellar radial-velocity variation induced by the
orbit.
% Results
We provide (1)~a catalog of \nunimodal\ companions for which the stellar
companion properties can be confidently determined in a posterior sense, (2)~a
catalog of \nhighK\ stars that likely have companions, but for which
(in most cases) more observations would be needed to uniquely determine the
orbital properties, and (3)~posterior samplings for all orbital-companion
parameters for all stars in the parent sample.
We highlight interesting systems and show the characteristics of stars with
confidently determined companion properties.
\end{abstract}

\keywords{
  binaries:~spectroscopic
  ---
  methods:~data~analysis
  ---
  methods:~statistical
  ---
  planets~and~satellites:~fundamental~parameters
  ---
  surveys
  ---
  techniques:~radial~velocities
}

\section{Introduction} \label{sec:intro}

Stars typically have companions.
Main sequence stars in the solar neighborhood more often appear in binary or
multiple star systems rather than as solitary stars (e.g.,
\citealt{Duquennoy:1991,Raghavan:2010,Tokovinin:2014}).
This is likely a generic outcome of star formation: For example, turbulent
fragmentation in collapsing protostellar clouds can produce stellar multiplets
within length-scales comparable to their spheres of influence (e.g.,
\citealt{Raskutti:2016}).
Binary and multiple star systems are therefore of great interest in
astrophysics: The population of stars and their companions encodes information
about star formation processes, stellar parameters and evolution, and the
dynamics of multi-body systems (for recent reviews, see
\citealt{Duchene:2013,Moe:2017}).

Most of what is known about stellar companions comes from studies of
nearby main-sequence (MS) stars.
MS stars with companions have a large dynamic range of constituent and orbital
characteristics.
For example, binary stars have mass-ratios that span from $\approx 0.03$
to 1 (e.g., \citealt{Kraus:2008}), and have periods from days to
millions of years (e.g., \citealt{Raghavan:2010}).
\todo{multiplicity in general: occurrence rates}

Less is known about non-interacting or detached companions to evolved stars.
Such systems are interesting because they enable studying extreme outcomes or
conditions of star formation and multiplicity.
For example, evolved stars are typically much brighter than their MS
counterparts, which enables characterizing the stellar companion population
throughout the Galactic disk over a wide range of chemical abundances.
They also provide a means to study the outcomes of stellar engulfment or
common-envelope evolution: Do short-period companions exist around evolved
stars, and do their statistics constrain stellar evolution on the red giant
branch (RGB)?
\todo{motivate time-domain spectroscopy}

Time-domain radial-velocity measurements of stars contain information about
massive companions: even with two successive observations of a single star, a
difference in the measured radial velocities implies the existence of at least
one companion.
However, with few or imprecise radial-velocity measurements, the orbital
properties of the companion(s) are very poorly constrained (e.g.,
\citealt{Price-Whelan:2017}).
The vast majority of spectroscopic targets with repeat observations in the
largest (by number of objects) stellar spectroscopic surveys are often observed
just a few times with sparse, non-uniform phase coverage.
Most prior searches for companions using survey RV data have therefore
restricted their searches to only sources with many, high-precision epochs, so
that the orbital solution can be unambiguously determined (e.g.,
\citealt{Troup:2016}), or have used simple statistics computed from the data to
study multiplicity (e.g., $\textrm{RV}_\textrm{max}$; \citealt{Badenes:2017}).

If there are only a few radial-velocity measurements made per star, and the
companion spectrum is not observed, any measured radial-velocity values will be
consistent with many different combinations of (primary) orbital parameters
(period, amplitude, eccentricity, etc.).
To identify companions to the typical star observed in a spectroscopic survey,
we therefore face at least one major challenge: how, given a small number of
observations of a primary star, do we reliably obtain posterior information
about the binary-system properties?
In general, the likelihood function---and the posterior probability distribution
function (pdf) under any reasonable prior pdf---will be highly multimodal, and
many of the modes will have comparable integrated probability density.
For example, with just two radial-velocity measurements, a harmonic series of
period modes will exist in the likelihood function.

We have solved this problem previously, though with limitations (to be discussed
more below), with \thejoker\ (\citealt{Price-Whelan:2017}).
\thejoker\ is a Monte Carlo rejection sampler that is computationally expensive
but probabilistically righteous:
It delivers independent posterior pdf samples for single-companion binary
orbital parameters, given any number of radial-velocity measurements.
Here we use \thejoker\ to generate posterior pdf samples for stars observed by
the \apogee\ survey (see \sectionname~\ref{sec:data}; \citealt{Majewski:2017}).

The \apogee\ surveys primarily target red-giant-branch (\RGB) stars, which are
ideal for the study of single-line binary systems.
For one, because they are so luminous, they are unlikely (in general) to have
equally-bright companions, and their spectra are therefore well-approximated or
fit as single-line objects.
When this constraint is not met, \thejoker\ will in general fail, and a model
that fits for a mixture of stellar spectra is more appropriate (e.g.,
\citealt{Kareem}).
The subset of \RGB\ stars in the ``red clump'' (\RC) are even more powerful as
they are standard candles and have masses that can be estimated using
spectroscopy (using dredged-up elements; \citealt{Martig:2016,Ness:2016}).
With primary-star mass estimates, the binary-orbit fitting will return
$m_2\,\sin i$ (minimum mass) estimates for the secondary, and not just estimates
of the so-called ``binary mass function.''
Additionally, the \apogee\ pipelines (\citealt{Garcia-Perez:2016}) and also
\thecannon\ (\citealt{Ness:2015}) produce detailed abundance estimates for \RGB\
and \RC\ stars.
If there are causal relationships between chemical abundances and binary
companions---as are expected---these should be measurable.

By making cuts on this library of posterior pdf samples (described in detail in
\sectionname~\ref{sec:whatever}), we deliver a catalog of binary-star systems
from the \apogee\ survey \todo{with ... (tease some statistics)}.
We additionally highlight the subset of this catalog with \RC\ star primaries,
for which we can compute companion mass and pericentric distance.

\section{Data} \label{sec:data}

All data used in this work comes from the publicly-available data release 14
(\DR) of the \apogee\ survey (\citealt{Majewski:2017,Abolfathi:2017}), a
component of the Sloan Digital Sky Survey IV (\sdssiv;
\citealt{Gunn:2006,Blanton:2017}).
\apogee\ is designed to map stars across much of the Milky Way by obtaining
high-resolution ($R \sim 22,500$) infrared ($H$-band) spectroscopy of primarily
\RGB\ stars.
Targets are selected with simple color and brightness cuts, but the survey uses
fiber-plugged plates with a maximum of 300 fibers per each $\approx
1.5~\textrm{deg}^2$ field of view, leading to ``pencil-beam''-like sampling of
the stellar distribution.
In order to meet signal-to-noise ratio requirements, most \apogee\ stars are
observed multiple times in a series of ``visits,'' typically with at least one
visit separated by a month or more in order to help identify binary stars.

Data taken as part of the \apogee\ survey is reduced with a multi-step data
reduction pipeline that ultimately solves for the stellar parameters, chemical
abundances, and radial velocities for each target
(\citealt{Nidever:2015}).
Most relevant for this work, the visit radial velocities (RVs) are determined
using an iterative scheme: the individual visit spectra are combined using
initial guesses for the relative RVs into a coadded spectrum, which is then used
to re-derive the relative visit velocities.
The stellar parameters---surface gravity, $\log g$, and effective temperature,
$T_{\textrm{eff}}$---and the chemical abundances are determined from the coadded
spectrum as a part of the \apogee\ Stellar Parameters and Chemical Abundances
Pipeline (\acronym{ASPCAP}; \citealt{ASPCAP}).

% Notebook: "Numbers of visits and sample overview"
\begin{figure}[h]
\begin{center}
\includegraphics[width=0.7\textwidth]{nvisits.pdf}
\end{center}
\caption{%
Number of \apogee\ stars in logarithmic bins of number of visits that pass the
quality cuts described in \sectionname~\ref{sec:data}.
In total, this work uses \nvisits\ visits and \nstars\ unique sources.
\label{fig:nvisits}
}
\end{figure}

We use the primary data products from \apogee\ \DR\ (i.e. the \texttt{allStar}
and \texttt{allVisit} files) which contain 258,475 unique source IDs
(\texttt{APOGEE\_ID}) and 1,054,381 unique visits.
We select all stars with $\geq 3$ visits that each pass a set of quality cuts,
described below.
For each visit, we require that the visit velocity uncertainty is $< 100~\kms$
(\texttt{VRELERR}) and the following bits are not set in the \texttt{STARFLAGS}
bitmask: \texttt{PERSIST\_HIGH}, \texttt{PERSIST\_JUMP\_POS},
\texttt{PERSIST\_JUMP\_NEG}, \texttt{VERY\_BRIGHT\_NEIGHBOR}, \texttt{LOW\_SNR}.
For each star, we require that $0 < \log g < 4$ and the following bits are not
set in the \texttt{ASPCAPFLAGS} bitmask: \texttt{STAR\_BAD}.
After these cuts, and the requirement of $\geq 3$ visits for a given star, we
are left with \nvisits\ visits for \nstars\ unique sources.
\figurename~\ref{fig:nvisits} shows the number of stars in several bins of
number of visits that pass the above quality cuts: \todo{XX}\% of the stars in
\DR\ have $< 8$ visits.
\figurename~\ref{fig:loggteff} ...

% Notebook: "Numbers of visits and sample overview"
\begin{figure}[h]
\begin{center}
\includegraphics[width=\textwidth]{logg_teff_feh.pdf}
\end{center}
\caption{%
\textit{Left panel}: Number of stars in bins of iron abundance,
$[\textrm{Fe}/\textrm{H}]$, that pass the quality cuts described in
\sectionname~\ref{sec:data}.
\textit{Right panel}: Distribution of stars in our sample in stellar parameters
log-surface-gravity, $\log g$, and effective temperature, $T_{\textrm{eff}}$,
with points colored by the iron abundance.
\label{fig:loggteff}
}
\end{figure}


\section{Methods}

\subsection{Orbit inference and velocity modeling}
\label{sec:fitting}

For every source in the sample of \apogee\ stars defined in
\sectionname~\ref{sec:data}, we obtain a posterior sampling in binary-system
parameter space, treating it as a single-lined (SB1) spectroscopic binary system
with a single companion.
This sampling is performed under a relatively uninformative prior pdf, and the
resulting posterior samplings are used to discover and characterize individual
binary-star systems and generate a catalog of companions.
We use \thejoker\ to perform the posterior samplings; we briefly describe the
algorithm below, but a full description is given in previous work
(\citealt{Price-Whelan:2017}).
In this \sectionname, we describe the assumptions and method used to generate
individual-system samplings.

We perform our individual system fits---that is, make our posterior
samplings---under the following assumptions:
\begin{description}
\item[no multiplets] All radial-velocity variations of the primary star are
  induced by a single companion.
  This is motivated by the idea that triple star systems are usually
  hierarchical so that the period of the inner binary is typically much shorter
  than the orbital period of the outer body (\todo{CITATION NEEDED}).
  At present, we ignore the possibility of higher-order multiple systems.
  % \todo{revisit this - do we want to also sample over a long-term trend?}
\item[Kepler] Related to the first point, all velocity variations of the primary
  are gravitational, and we therefore ignore the possibility of coherent intrinsic variation from, e.g., stellar oscillations.
\item[SB1] All spectra are single-lined; that is, we assume that the secondary
  is significantly fainter and is thus undetected in the spectra.
  This assumption is motivated by the fact that we expect \RGB\ stars to be
  substantially more luminous than their typical companion.
  However, there are known main-sequence double-lined binary stars in the
  \apogee\ catalog (\citealt{El-Badry:2018}), and an expected but unknown
  fraction of \RGB--\RGB\ binaries.
\item[simple noise model] Measurements are unbiased and noise estimates are
  correct up to an unknown excess variance.
  All noise contributions result in Gaussian uncertainties on the individual
  radial-velocity measurements.
\end{description}

\thejoker\ is a custom-built Monte Carlo sampler designed to produce independent
posterior samples in Keplerian orbital parameters, given radial-velocity
measurements under the assumptions listed above.
Our parametrization of the orbital elements is similar to the notation in
\citet{Murray:2010}:
The radial velocity $v$ at time $t$ is given by
\begin{equation}
  v(t;\bs{\theta}) = v_0 + K\,[\cos\left(\omega + f(t; e, P, M_0, t_0)\right) +
    e\,\cos\omega]
\end{equation}
where $\bs{\theta} = (P,e,M_0,\omega,K,v_0)$---period, eccentricity, mean
anomaly at a reference time $t_0$, argument of pericenter, velocity
semi-amplitude, Barycentric velocity---and the true anomaly, $f$, is a function
of time and the specified parameters (see \sectionname~2 of
\citealt{Price-Whelan:2017} or \eqname~63 in \citealt{Murray:2010}).
In addition to the orbital parameters listed above, \thejoker\ can also generate
samples in an ``excess variance'' parameter, $s^2$, that is added to the
per-visit measurement variances.
This parameter allows us to test whether the visit RV uncertainties are
underestimated
For upper \RGB\ stars the inferred excess variance will be a combination of
extra systematic uncertainty and true astrophysical surface jitter
(\todo{CITATION NEEDED}).

\thejoker\ was designed for the extremely multi-modal pdfs expected when the
number of radial-velocity measurements of a source is small, or the data are
sparse (in phase-coverage) or noisy.
While other Markov Chain Monte Carlo (MCMC) methods have difficulty producing
independent samples with such data, \thejoker\ succeeds by brute force:
After generating an initial (very large) library of prior samples from an
assumed prior pdf (see below), the (typically multi-modal) likelihood is
evaluated at each sample and used to rejection sample.
In practice, given a number of requested samples for each star, the sampling
proceeds iteratively: since it is easier to accept samples when the data is
sparse or noisy, far more prior sample draws (and thus likelihood evaluations)
must occur under very constraining data.

% \subsection{Model selection}
%
% In addition to generating posterior samples for each source, we also compute the
% fully marginalized likelihood (FML) under two other models: (1) a model in which
% the radial velocity of the source is constant, and (2) a model in which the
% radial-velocity variations are purely linear.
% We later use these FML values to define cuts on our full catalog of posterior
% samples to produce a catalog of companions.
%
% \todo{equations?}
%
% Examples of outputs and etc.

\subsection{Individual-system posterior samplings}
\label{sec:samplings}

Here we describe the procedure we use to generate per-target posterior samplings
for the \apogee\ targets.
We execute the full procedure twice for different goals (as described in
\sectionname~\ref{sec:catalogs}), and only here outline the key steps in the
pipeline.

For all \nstars\ \apogee\ stars with $\geq 3$ good visits (see
\sectionname~\ref{sec:data}), we use \thejoker\ to generate posterior samplings
for each star under the assumptions listed above (see
\sectionname~\ref{sec:fitting}).
We start by generating a library of \nprior\ prior samples generated under a
prior similar to that defined in \citet{Price-Whelan:2017}:
\begin{itemize}
    \item uniform or isotropic in angle parameters,
    \item uniform in log-period over the domain $[1,32768]~\textrm{day}$,
    \item a beta distribution over eccentricity with fixed parameters (\citealt{Kipping:2013}).
\end{itemize}
For the prior over the excess variance parameter, $s^2$, we initially use
a Gaussian over the transformed parameter $y = \ln s^2$ with the mean and
standard deviation $(\mu_y, \sigma_y)$ indicated where the runs are described
(see \sectionname~\ref{sec:catalogs}).
The reference time for each star is set to the minimum visit epoch; $M_0$ then
becomes the mean anomaly at the first visit observation.
\tablename~\ref{tbl:params} contains descriptions of all parameters and priors
used.

\begin{table}[h]
    \centering
    \begin{tabular}{ r l l }
    \hline
    name & prior & description \\
    \hline
    $P$ & $\ln P \sim \mathcal{U}(1, 32768)~\textrm{day}$ & period \\
    $e$ & $e \sim \textrm{Beta}(0.867, 3.03)$ & eccentricity \\
    $t_0$ & fixed & reference time \\
    $M_0$ & $M_0 \sim \mathcal{U}(0, 2\pi)~\textrm{rad}$ & mean anomaly at reference time \\
    $\omega$ & $\omega \sim \mathcal{U}(0, 2\pi)~\textrm{rad}$ & argument of pericenter \\
    $s^2$ & $\ln s^2 \sim \mathcal{N}(\mu_y, \sigma_y^2)$ & extra variance added to each visit variance \\
    $K$ & $\mathcal{N}(0, \sigma_{v}^2)~\kms$ & velocity semi-amplitude \\
    $v_0$ & $\mathcal{N}(0, \sigma_{v}^2)~\kms$ & system barycentric velocity \\
    \hline
    \end{tabular}
    \caption{Summary and description of parameters. $\textrm{Beta}(a, b)$ is the
    beta distribution with shape parameters $(a, b)$, $\mathcal{U}(a, b)$ the
    uniform distribution over the domain $(a, b)$, and $\mathcal{N}(\mu,
    \sigma^2)$ is the normal distribution with mean $\mu$ and variance
    $\sigma^2$.
    For the systemic velocity and semi-amplitude, $(v_0, K)$, we use a broad
    Gaussian prior that is formally inconsistent between \thejoker\ and
    follow-up MCMC sampling (see \sectionname~\ref{sec:mcmc}): in \thejoker\ we
    assume Gaussian priors with $\sigma_v$ much larger than the measurement
    uncertainty so they can be neglected ($\sigma_v \approx \infty$), whereas
    when running MCMC we fix $\sigma_v = 10^3~\kms$.}
    \label{tbl:params}
\end{table}

We request \nposterior\ posterior samples for each source.
Depending on the data quality and phase coverage of the visits, \thejoker\ will
require different numbers of prior samples in order to rejection sample down to
the requested number of posterior samples: For few-epoch or noisy RV data, many
prior samples will pass the rejection step, whereas for very precise or
many-epoch RV data, \thejoker\ may need to go process the full library of prior
samples.
We therefore generate the posterior samples using an iterative procedure that
adaptively predicts how many prior samples to test for each star.
For sources with very constraining data, \thejoker\ may return fewer than the
requested number of samples (as few as one sample).
When this occurs, we continue sampling either using standard MCMC, or by
increasing the size of the prior cache and continuing rejection sampling with
\thejoker.


\subsubsection{``Needs MCMC'': Following up \thejoker\ with MCMC}
\label{sec:mcmc}

If just one posterior sample is returned after exhausting the full library of
prior samples, or if multiple (but fewer than \nposterior) are returned that all
lie within a single mode of the posterior pdf, the posterior pdf over orbital
parameters is treated as effectively unimodal: these stars are flagged ``needs
MCMC.''
In this case, we use the location of the returned sample (if only one is
returned), or a randomly chosen sample from those returned (if multiple samples
are returned within one mode) to generate a small Gaussian ball of initial
conditions and use standard MCMC to continue sampling until we obtain
\nposterior\ samples.

We use an ensemble MCMC sampling algorithm (\citealt{Goodman:2010}) implemented
in \python\ (\package{emcee}; \citealt{Foreman-Mackey:2013}) to perform the
samplings.
We transform the standard Keplerian orbital parameters to a safer
parametrization, $(\ln P, \sqrt{K}\,\cos M_0, \sqrt{K}\,\sin M_0, \sqrt{e}\,\cos
\omega, \sqrt{e}\,\cos \omega, \ln s^2, v_0)$, to sample in.
This reparametrization is safer and more efficient for sampling with
\package{emcee}, which expects parameters to be components of a vector so that
linear operations can be applied (see, e.g., \citealt{Hogg:2017}); the angle
variables $(\omega, M_0)$ don't meet this requirement in the standard
parametrization.
We use the same prior pdfs as in \thejoker\ when running MCMC (see
\tablename~\ref{tbl:params}).

For each star that is flagged ``needs MCMC,'' we run \package{emcee} with 1024
walkers for 16384 steps, take the final walker positions, and downsample at
random until we have \nposterior\ samples.
We compute the Gelman--Rubin convergence statistic, $\hat{R}_j$,
(\citealt{Gelman:1992}) for each parameter $j$ and include these values in the
catalogs below when standard MCMC is run.
We also provide a binary flag, ``converged,'' for each sampling continued with
MCMC that is set to true if:
\begin{equation}
\underset{j}{\textrm{mean}}\left(\hat{R}_j\right) < 1.1 \quad .
\end{equation}


\subsubsection{``Needs more prior samples'': Continuing \thejoker\ sampling}

If more than one posterior sample is returned after exhausting the full library
of prior samples, and the samples lie in multiple modes of the posterior pdf,
the only way to proceed is to generate more prior samples and continue running
\thejoker: these stars are flagged ``needs more prior samples.''
In this case, we generate another equal-sized library of prior samples (a total
of $2\times\nprior$ samples) and re-do the rejection sampling.
We note that because of the way the rejection sampling step is done, this is not
equivalent to concatenating the results from a second, independent run of
\thejoker: the log-likelihood values for all of the prior samples must be used.
If at the end of this second run the target still has fewer than \nposterior\
samples, the sampling is flagged as ``incomplete.''

\subsection{Null control sample}
\label{sec:control-sample}

We construct a control sample of simulated data with no RV variability to assess
our false-positive rates in the selections below (see
\sectionname~\ref{sec:catalogs}).
We randomly pick \ncontrol\ stars from the parent sample used in this work and
replace the visit velocity measurements with simulated data.
For each star $n$, we randomly sample an excess variance parameter value from
the prior, $s_n$.
For each visit $k$, we then sample a new velocity $v_{nk}$ by drawing from a
Gaussian with mean equal to the mean of the real data visit velocities,
$\bar{v}_n$, and variance equal to the sum of the visit variance (uncertainty),
$\sigma_{nk}$, and the excess variance,
\begin{equation}
    v_{nk} \sim \mathcal{N}(\bar{v}_n, \sigma_{nk}^2 + s_n^2) \quad .
\end{equation}
When we save the control data, we only store the visit uncertainty and
``forget'' the fact that the simulated data is generated with excess variance.


% \subsection{Implementation notes}
%
% \todo{Nontrivial computation - details about running on cluster, TheJoker vs. emcee}

% How do we run TheJoker on all of APOGEE? Run on a cluster, send batches of samples to processors because individual likelihood calls are fast
%
% How do we do MCMC? Similar issues as above, but less control because not inherently embarassingly parallel


\section{Experiment: infer the excess variance distribution}
\label{sec:inferjitter}

As an initial use of the per-source posterior samplings, we use a hierarchical
Bayesian model to infer the parameters of the (assumed Gaussian) prior over the
log-excess-variance parameter, $(\mu_y, \sigma_y)$ (see
\tablename~\ref{tbl:params}).
This inference serves as a test-case for future work, where we intend to use the
independent posterior samplings to construct a hierarchical inference over
companion population properties.
This is also a test of the visit velocity uncertainties reported in the \apogee\
data products: If the catalog uncertainties are significantly underestimated, we
expect the inferred log-excess-variance distribution parameters to tend towards
larger values.

In detail, we maximize the marginal likelihood of the population-level
parameters $(\mu_y, \sigma_y)$.
We compute this marginal likelihood using the per-object posterior samples
re-weighted by the ratio of the value of the hyperprior evaluated at a given,
new set of parameters $(\mu_y, \sigma_y)$ over the value of the default prior at
the previously assumed values, $(\mu_{y,0}, \sigma_{y,0})$ (see above).
This trick has been used in other hierarchical inferences as a way to
marginalize over the per-object parameters to infer population-level parameters
(\citealt{Hogg:2010,Foreman-Mackey:2014}); we describe how to compute the
marginal likelihood in detail in \sectionname~\ref{sec:hierarch}.

We execute a full run of \thejoker\ on all \nstars\ \apogee\ stars in our parent
sample using initial values for the excess variance distribution parameters
chosen so that $\sqrt{e^{\mu_{y,0}}} \approx 200~\mps$: $(\mu_{y,0},
\sigma_{y,0}) = (10.6, 3)$ in units of $\mps$.
This run took approximately 300 hours on a compute cluster with 448 cores, with
the time dominated by sources with many ($\gtrsim 10$) visits.  For this initial
run, we do not follow up on stars that return $<$\nposterior\ samples.

We use 1,825 stars with $>10$ visits and $\log g < 2$ (to avoid upper RGB stars
that have large intrinsic jitter) and maximize the above likelihood to determine
better hyperparameters for the log-excess-variance parameter distribution.
\figurename~\ref{fig:infer-jitter} shows the distribution corresponding to the
maximum-likelihood hyperparameters, $\alpha^* = (\mu_y^*, \sigma_y^*) = (9.50,
1.64)$.
These values are consistent with the estimated systematic floor of the visit
velocity uncertainties of $\approx 100$--$200~\mps$ estimated from stars
observed multiple times and on multiple plates (\citealt{Nidever:2015}).

However, there are a number of caveats to keep in mind about this estimate of
the excess variance distribution.
First, we do not remove triple or other multiple systems: For any individual
system, the radial velocity variations induced by other massive bodies will lead
to larger preferred values for the excess variance parameter.
We don't expect there to be a large number of triple systems in our sample, but
this is still an important consideration for future efforts.
Second, we are sensitive to outliers and very non-Gaussian systematic error
distributions.
We later assess the non-Gaussianity of the visit velocity uncertainties by
looking at the visit velocity residuals away from the orbit samples produced by
\thejoker\ using the updated excess variance distribution (see
\sectionname~\ref{sec:discuss-assumptions}).

% Notebook: "Infer jitter"
\begin{figure}[h]
\begin{center}
\includegraphics[width=\textwidth]{infer-jitter}
\end{center}
\caption{%
Inferred prior over excess variance parameter ($y = \ln s^2$) using posterior
samples for 1,825 lower-RGB stars with $>10$ visits.
Left and right panels show the distribution corresponding to the
maximum-likelihood parameters $(\mu_y, \sigma_y) = (9.50, 1.64)$ in log and
linear, respectively.
\label{fig:infer-jitter}
}
\end{figure}


\section{Companion catalogs}
\label{sec:catalogs}

Using the maximum-likelihood hyperparameters derived from the initial posterior
samplings (see \sectionname~\ref{sec:inferjitter}), we update and fix the
excess variance prior distribution parameters---$(\mu_{y}, \sigma_{y}) = (9.50,
1.64)$, in units of $\textrm{m}~\textrm{s}^{-1}$---and re-run \thejoker\ on the
parent sample of \apogee\ \DR\ stars.
This methodology, in which we estimate a prior from the data and then fix it, is
typically referred to as ``empirical Bayes''; In this case, it is an
approximation to doing a hierarchical inference of the individual system orbital
parameters and the excess variance distribution hyperparameters simultaneously.

From this run, 91,096 stars completed and successfully returned \nposterior\
samples using \thejoker\ alone.
The remaining 5,135 stars did not return \nposterior\ samples: 4,744 were
flagged as \emph{needs more prior samples}, 391 as \emph{needs MCMC}.
We continue generating samples for these stars using the methodology explained
above (see \sectionname~\ref{sec:samplings}).
In the end, \todo{XX} stars have completed samplings with \nposterior\ samples,
and, after re-running \thejoker\ with twice as many prior samples, \todo{XX}
stars still have fewer samples that are multi-modal in period.

The full catalog of posterior samples for all \apogee\ stars in our parent
sample is available online.
\todo{Structure of table, where to download, make a webpage with some
information?}

We also run on the null control sample (see \sectionname~\ref{sec:control}) with
the same parameters.
From the control sample run, 13 ``stars'' are flagged as \emph{needs MCMC}, and
1023 are flagged as ``needs more prior samples'';
For these stars, we don't continue sampling with \thejoker\ or MCMC and only use
the posterior samples returned from the \thejoker.
% See: Sample K cuts.ipynb


\subsection{Stars with companions}
\label{sec:catalog-confident}

We don't expect there to be a sharp transition in orbital parameters between
stars with companions and stars without companions: There is a continuum of
companion properties.
For example, the companion masses can be low mass, or at long periods, or at
high inclination, all of which will make the velocity semi-amplitude, $K$, small
for a given system.
Therefore, from the radial velocity data alone, there is no way to define a
simple cut to select a complete sample of stars with companions.
It is nevertheless possible to define a cut that selects stars with
high-confidence companions.

We select a sample of stars that confidently have companions using percentiles
computed from the posterior samples in the velocity semi-amplitude parameter,
$K$.
\figurename~\ref{fig:lnK-percentiles}, shows the distribution of 1st percentiles
of $\ln K$ computed for all stars with posterior samplings from \thejoker\
(filled, blue histogram), along with the same for posterior samplings for the
null control sample (solid, dark line).
To select stars with companions, we use a cut in the 1st percentile of the $\ln
K$ samples such that $<1\%$ of the control sample is selected, i.e. our
estimated false positive rate is $<1\%$.
We use a threshold of $\ln K = 0$ to meet this constraint (vertical, dashed line
in \figurename~\ref{fig:lnK-percentiles}); \nhighK\ stars pass this cut.

% Notebook: "Sample K cuts.ipynb"
\begin{figure}[h]
\begin{center}
\includegraphics[width=\textwidth]{lnK-percentiles}
\end{center}
\caption{%
Distribution of 1st percentiles in $\ln K$ (in units of \kms) for the parent
\apogee\ sample (blue, filled) and for the control sample (solid, dark line).
The vertical line (dashed, orange) indicates our adopted cut to select high-$K$
stars that likely have companions; $<1\%$ of the control sample falls above this
cut.
\label{fig:lnK-percentiles}
}
\end{figure}

\tablename~\ref{tbl:} contains a subset of the \apogee\ \DR\ \texttt{allStar}
catalog information for the stars that pass the above cut and therefore likely
have companions; For brevity below, we refer to this sample as the ``high-$K$''
stars, and the complimentary sample as the ``low-$K$'' stars.
\figurename s~\ref{fig:highK-0} and \ref{fig:highK-1} show eight examples of
high-$K$ stars, i.e. stars that likely have companions, with different numbers
of visits, $N$, indicated on the panels.
\figurename s~\ref{fig:lowK-0} and \ref{fig:lowK-1} is the same for eight
low-$K$ stars, i.e. stars that either have no companions, low-mass companions,
or companions at long periods or high inclination.

The majority of the stars in the high-$K$ sample have poorly constrained orbital
parameters because the posterior pdfs are very multimodal.
The high-$K$ sample includes all stars in the catalogs described in the next two
subsections, however these two catalogs are mutually exclusive with one another.

% Notebook: "HighK"
\begin{figure}[hp]
\begin{center}
\includegraphics[width=\textwidth]{highK-0}
\end{center}
\caption{%
Examples of stars in our high-$K$ sample, i.e. stars that likely have companions
(see \sectionname~\ref{sec:catalog-confident}), with different numbers of
visits.
Left panels show the data (black markers, error bars are the visit velocity
uncertainties) with 128 random orbits from the \nposterior\ posterior samples
under-plotted (lines, blue); The \texttt{APOGEE\_ID} of each target and the
number of visits that pass our quality cuts, $N$, are indicated on each panel.
Right panels show the \nposterior\ posterior samples visualized in
period--eccentricity space.
In most cases, despite confidently having companions based on the RV amplitude,
the permitted orbit fits are highly multimodal.
\label{fig:highK-0}
}
\end{figure}

\begin{figure}[hp]
\begin{center}
\includegraphics[width=\textwidth]{highK-1}
\end{center}
\caption{%
Continuation of \figurename~\ref{fig:highK-0}.
\label{fig:highK-1}
}
\end{figure}

\begin{figure}[hp]
\begin{center}
\includegraphics[width=\textwidth]{lowK-0}
\end{center}
\caption{%
Same as \figurename~\ref{fig:highK-0}, but for stars in the low-$K$ sample, i.e.
stars that have undetected or no companions.
\label{fig:lowK-0}
}
\end{figure}

\begin{figure}[hp]
\begin{center}
\includegraphics[width=\textwidth]{lowK-1}
\end{center}
\caption{%
Continuation of \figurename~\ref{fig:lowK-0}.
\label{fig:lowK-1}
}
\end{figure}


\subsection{Confident companions with highly constrained orbits}
\label{sec:catalog-multimodal}

A subset of the high-$K$ sample (\sectionname~\ref{sec:catalog-confident}) have
multimodal posterior distributions over orbital parameters that are limited to
just a few qualitatively different solutions.
For these stars, one or a few more well-timed RV measurements would likely lead
to uniquely-determined orbital parameters.
\todo{How exactly do we select these stars?}

\figurename~\ref{fig:todo} shows a few examples of stars that meet these
criteria.
Note that in these cases, there are certain times at which a future observation
would be far more informative: For example, \todo{top panel, time X vs. time Y}.

% % Notebook: TODO
% \begin{figure}[hp]
% \begin{center}
% \includegraphics[width=\textwidth]{highK-unimodal}
% \end{center}
% \caption{%
% TODO
% \label{fig:highK-unimodal}
% }
% \end{figure}

\todo{Describe catalog contents: Also include M2min and primary mass when
possible?}


\subsection{Confident companions with effectively uniquely determined orbits}
\label{sec:catalog-unimodal}

\todo{Also include M2min and primary mass when possible?}

% Notebook: HighK-unimodal-catalog 
\begin{figure}[hp]
\begin{center}
\includegraphics[width=\textwidth]{highK-unimodal}
\end{center}
\caption{%
Examples of high-$K$ stars with effectively unimodal posterior samplings and
converged MCMC samplings.
Each panel shows the data (black markers), phase-folded at the period of the
posterior sample with maximum posterior probability.
Visit velocity uncertainties are shown as black error bars (these are typically
smaller than or equal to the size of the markers), and the inferred jitter as
grey, capped error bars.
Line (blue) shows the orbit compute from the posterior sample with maximum
posterior probability.
\label{fig:highK-unimodal}
}
\end{figure}


\section{Results}
\label{sec:results}

\subsection{Red clump stars lack short-period companions}

\todo{Period-logg plot - lack of short-period companions for red clump stars?}

Red clump (\RC) stars are horizontal branch stars and have therefore already
climbed the RGB, where the surface of the star can grow to be several AU in
size.
\RC\ stars are therefore expected to less frequently have companions at short
periods ($P<XX$~day) because the companions would have been engulfed during the
RGB phase.
\figurename~\ref{fig:todo} hints that there is an apparent dearth of
short-period companions around $\log g \sim 2.5$ (the approximate location of
the red clump).
Another way to look at this is to compare the fraction of subgiant stars with
short-period companions to the same for stars near the \RC.

\todo{To assess this, take boxes in log g-Teff}
We find that, empirically, XX\% of the subgiants and YY\% of the \RC\ stars have
companions with $XX < P < XX$~day.
\todo{Assuming selection function doesn't preferentially bias...}


\subsection{Evidence for tidal circularization at $P\sim 10~\textrm{day}$}

\todo{Expectation from theory? What's special at 10 days?}

\figurename~\ref{fig:unimodalPe} shows the orbital period and eccentricity...

\todo{Period - eccentricity plot? Tidal circularization (See "Debug MCMC
continue")}

% Notebook: HighK-unimodal
\begin{figure}[h]
\begin{center}
\includegraphics[width=0.7\textwidth]{unimodal-P-e}
\end{center}
\caption{%
TODO:
\label{fig:unimodalPe}
}
\end{figure}


\subsection{Companion masses and mass ratios}

\todo{Companion mass and mass-ratio distribution for stars with Ness / Martig masses}


\subsection{Impossible companions: coherent oscillations?}

\todo{The companions that don't make sense / can't be}


\section{Discussion}

\todo{comment on short period shit, eccentricities}

\subsection{Assumptions}
\label{sec:discuss-assumptions}

It is important to keep in mind that the companion catalogs and results
presented in this \documentname\ depend on the assumptions laid out in
\sectionname~\ref{sec:fitting}.
We have only considered radial velocity modulations from a single massive
companion---\emph{no multiplets}; This is a fundamental limitation of our search
and methodology.
However, stars with multiple companions will likely be included in our companion
catalog anyway, only with two-body orbital solutions that either don't fit the
data well or pick out the shortest period orbit.

Related to the above, we assume that all RV variations are
gravitational---\emph{Kepler}.
We see tentative evidence for coherent oscillations at the upper giant branch
(see \figurename~\ref{fig:todo}), which we presently treat as excess variance or
intrinsic jitter.
If these oscillations are indeed from asteroseismic modes, \todo{why is this
interesting?}.

We assume that one star in each two-body system overwhelmingly dominates the
luminosity and therefore spectrum of each system---\emph{SB1}.
As seen in \figurename~\ref{fig:todo}, there are some companions with minimum
masses comparable to or consistent with being larger than the masses of the
observed star.
These systems are either (a) nearly edge-on MS--RGB binaries (i.e. consistent
with our assumption), (b) SB2 systems where the \apogee\ pipeline failed to flag
the source as having broad lines or a bad fit, or (c) RGB--stellar remnant
systems with a black hole or neutron star companion.
A subset of these systems that look like they fall in category (c) are being
followed-up to obtain further RV measurements to test whether any companions are
black holes (Geha et al., in prep.).

Finally, we have assumed that the visit velocity error distribution is Gaussian,
and that the visit velocity uncertainties could be
under-estimated---\emph{simple noise model}.
We can test this assumption using posterior samples from \thejoker\ by computing
the residuals away from our best-fit two-body orbital solutions.
We find that the normalized residuals appear to be very close to Gaussian over a
large dynamic range of normalized residual values, indicating that this
assumption may be sufficient.
\figurename~\ref{fig:residuals} shows the distribution of normalized residuals,
$R_{nk}$, for each visit: the $k$ visit velocities for each $n$ star, $v_{nk}$,
minus the predicted radial velocity from the best-fitting sample returned by
\thejoker, $\hat{v}^*_{nk}$, normalized by the excess-variance-included
uncertainty, $\sigma_{nk}^* = \sqrt{\sigma_{nk}^2 + \hat{s}_{n}^2}$, where
$\hat{s}_{n}$ is computed from the excess-variance parameter of the best-fitting
sample:
\begin{equation}
    R_{nk} = \frac{v_{nk} - \hat{v}^*_{nk}}{\sqrt{\sigma_{nk}^2 +
    \hat{s}_{n}^2}} \quad . \label{eq:normresid}
\end{equation}

% Notebook: "Residuals-vs-metadata"
\begin{figure}[h]
\begin{center}
\includegraphics[width=0.7\textwidth]{residuals}
\end{center}
\caption{%
Histogram shows the distribution of normalized visit residuals (see
\eqname~\ref{eq:normresid}) for all \nvisits\ used in the final run of
\thejoker\ (i.e. with updated excess variance distribution parameters inferred
in \sectionname~\ref{sec:inferjitter}).
The orange curve shows that expected for Gaussian uncertainties.
The distribution appears mostly Gaussian over a large range of residual values,
with slightly more populated tails and evidence of a few catastrophic outliers.
\label{fig:residuals}
}
\end{figure}


\subsection{Comparison to other APOGEE companion catalogs}
\label{sec:compare-troup}

There are at least two other recent catalogs of stellar systems and companions
based on \apogee\ data.

One of these studies focused on decomposing spectra of MS stars into mixtures of
stellar spectra (\citealt{El-Badry:2018}).
Conceptually, this method works because (a) for two unequal-mass stars,
unexpected absorption lines will appear superimposed on the brighter star's
spectrum, and (b) for close to equal-mass stars, the line depths and ratios will
not be well-matched by a single stellar model.
Using this technique, they identified thousands of candidate MS binaries and
trinaries, but did not consider giant stars ($\log g > 4$).
This sample is therefore complimentary to and non-overlapping with the catalog
presented in this work.

The other recent catalog searched for stellar and substellar companions to all
stars in \apogee\ \acronym{DR12} (including the RGB) that passed a series of
quality cuts, and had $\geq 8$ visits (\citealt{Troup:2016}).
For each star in the sample, orbits were fit to the visit RVs using a multi-step
orbit-fitting procedure: it starts by identifying significant periods and a few
harmonics of those periods, then fits a Keplerian orbit at each of these
harmonics using least-squares fitting (\citealt{De-Lee:2013}) with a modified
$\chi^2$ statistic that penalizes fits in which the phase coverage of the data
is poor.
This procedure is not guaranteed to provide a unique orbit solution.

Of the 382 companions released as a part of this previous search, only 188 of
the host stars passed the stellar parameter and quality cuts used to define the
parent sample in this work (see \sectionname~\ref{sec:data}).
We have looked at all of the overlapping stars to compare the previously derived
companion orbital properties to the posterior samplings derived with \thejoker.
We find that the comparisons fall in three categories:
(1) the parameters reported in \citet{Troup:2016} agree with the posterior
samplings, and the period distribution appears unimodal,
(2) the parameters reported in \citet{Troup:2016} identify one possible mode of
a likely multi-modal posterior pdf over orbital parameters, and
(3) the data changed significantly between \apogee\ \acronym{DR12} and \DR, so
no meaningful comparisons can be made; Roughly 1/3 of the comparison sample
falls into each class.
\figurename~\ref{fig:troup-unimodal} shows a few representative cases in which
the \citet{Troup:2016} orbital parameters (orbit shown as orange line in left
panels, parameters shown as orange + in right panels) is consistent with the
orbit samples from \thejoker.
\figurename~\ref{fig:troup-multimodal} shows a few representative cases in which
we find that the posterior pdf over orbital parameters is multimodal, and the
\citet{Troup:2016} orbit identifies one of these modes.
For completeness, \figurename~\ref{fig:troup-datachanged} shows two instances in
which the orbital parameters from \citet{Troup:2016} no longer make sense,
likely because the data changed between data releases.

\begin{figure}[h]
\begin{center}
\includegraphics[width=\textwidth]{unimodal.pdf}
\end{center}
\caption{%
TODO:
\label{fig:troup-unimodal}
}
\end{figure}

\begin{figure}[h]
\begin{center}
\includegraphics[width=\textwidth]{multimodal.pdf}
\end{center}
\caption{%
TODO:
\label{fig:troup-multimodal}
}
\end{figure}

\begin{figure}[h]
\begin{center}
\includegraphics[width=\textwidth]{data-changed.pdf}
\end{center}
\caption{%
TODO:
\label{fig:troup-datachanged}
}
\end{figure}

\subsection{Population inference}

Comment on things we could do with these samples hierarchical inference...

We are SB1 only, but should do SB2 as well - no reason to restrict.

\section{Conclusions}

All companion catalogs described in this work (\sectionname~\ref{sec:catalogs})
are available online at \todo{XX}.
The source code for this project is open source and available from
\url{https://github.com/adrn/TwoFace} under the MIT open source software
license.

\acknowledgements

It is a pleasure to thank Keith Hawkins (Columbia), \todo{who else?}.

The authors are pleased to acknowledge that the work reported on in this
paper was substantially performed at the TIGRESS high performance computer
center at Princeton University which is jointly supported by the Princeton
Institute for Computational Science and Engineering and the Princeton
University Office of Information Technology's Research Computing department.

\software{
    \package{Astropy} (\citealt{Astropy-Collaboration:2013}),
    \package{emcee} (\citealt{Foreman-Mackey:2013}),
    \package{IPython} (\citealt{Perez:2007}),
    \package{matplotlib} (\citealt{Hunter:2007}),
    \package{numpy} (\citealt{Van-der-Walt:2011}),
    \package{scipy} (\url{https://www.scipy.org/}),
    \package{schwimmbad} (\todo{JOSS paper}),
    \package{sqlalchemy} (\url{https://www.sqlalchemy.org/}),
    \package{thejoker} (\todo{thejoker zenodo?}),
    \package{twobody} (\todo{twobody zenodo?}).
}

\facility{\sdssiv, \apogee}

\clearpage

\bibliographystyle{aasjournal}
\bibliography{refs}

\appendix
\section{Hierarchical inference of the excess variance parameter}
\label{sec:hierarch}

For each $n$ of $N$ RGB stars in APOGEE, we obtain $K$ posterior samples over
primary orbital parameters $\bs{\theta} = (P, e, \omega, M_0, K, v_0)$ and the
excess-variance parameter, $y = \ln s^2$, using \thejoker; For brevity in
expressions below, we will use the vector
\begin{equation}
    \bs{w} = (\bs{\theta}, y)
\end{equation}
to represent the full set of parameters.
To obtain this sampling, we use an interim (Gaussian) prior on the
excess-variance parameter parametrized by a mean and standard deviation, i.e.
$\alpha_0 = (\mu_{y,0}, \sigma_{y,0})$ as described above.
For a given source, the posterior samples in the above parameters are drawn from
the distribution
\begin{equation}
    \bs{w}_k \sim p(\bs{w}_k \given D, \bs{\alpha}_0)
\end{equation}
where $D$ represents the data for a given object.

We want to compute the likelihood of all data from all $N$ stars, $\{D_n\}$,
given a new set of hyperparameters $\bs{\alpha}$
\begin{equation}
    p(\{D_n\} \given \bs{\alpha}) = \prod_n^N p(D_n \given \bs{\alpha})
\end{equation}
where in the above, we have assumed that this likelihood is separable ( the data
for each source are independent).
The per-source marginal likelihood in the above expression is given by
\begin{align}
    p(D_n \given \bs{\alpha}) &= \int \dd \bs{w}_n \, p(D_n \given \bs{w}_n) \,
      p(\bs{w}_n \given \bs{\alpha})\\
    &= \int \dd \bs{w}_n \, p(D_n \given \bs{w}_n) \, p(\bs{w}_n \given \bs{\alpha}) \,
      \frac{p(\bs{w}_n \given D_n, \bs{\alpha}_0)}{p(\bs{w}_n \given D_n, \bs{\alpha}_0)}\\
    &= p(D_n \given \bs{\alpha}_0) \, \int \dd \bs{w}_n \,
      \frac{p(\bs{w}_n \given \bs{\alpha})}{p(\bs{w}_n \given \bs{\alpha}_0)} \,
      p(\bs{w}_n \given D_n, \bs{\alpha}_0) \label{eq:marglike} \quad .
\end{align}
Using the Monte Carlo integration approximation, \eqname~\ref{eq:marglike} can
be simplified to a sum over prior value ratios of the $K$ posterior samples in
the log-excess-variance parameter for each $n$ star
\begin{equation}
    \approx \frac{\mathcal{Z}_n}{K} \,
      \sum_k^{K} \frac{p(y_{nk} \given \bs{\alpha})}{p(y_{nk} \given \bs{\alpha}_0)}
\end{equation}
where we have canceled the other priors (over $\bs{\theta}$), and all
normalization constants appear in the constant scale factor $\mathcal{Z}_n$.

The above expresion gives the marginal likelihood of the velocity data for a single source given new hyperparameters $\bs{\alpha}$.
The full marginal likelihood is then the product of these individual likelihoods
\begin{align}
    p(\{D_n\} \given \alpha) &\propto \prod_n^N \frac{1}{K} \,
      \sum_k^{K} \frac{p(y_{nk} \given \alpha)}{p(y_{nk} \given \alpha_0)}
      \quad .
\end{align}
In practice, we evaluate the log-marginal-likelihood
\begin{align}
    \ln p(\{D_n\} \given \alpha) &\propto \sum_n^N \left[
      \ln\left( \sum_k^{K} \frac{p(y_{nk} \given \alpha)}{p(y_{nk} \given \alpha_0)} \right)
      - \ln K\right]\label{eq:lnlike}\\
    &\propto \sum_n^N \left[
      \underset{k}{\textrm{logsumexp}}\left[ \ln{p(y_{nk} \given \alpha)} - \ln{p(y_{nk} \given \alpha_0)} \right]
      - \ln K\right]
\end{align}
where $\textrm{logsumexp}$ (the log-sum-exp trick) provides a more stable
estimate of the sum in \eqname~\ref{eq:lnlike}.

\end{document}
