\documentclass{article}

\usepackage{graphicx}
\usepackage{subfig}

\begin{document}

\section{}

\begin{figure*}[hp]
\begin{center}
\includegraphics[width=\textwidth]{figures/snr.pdf}
\end{center}
\caption{%
Signal-to-noise ratio (SNR) distribution for DR13 visits that also appear in
DR14.
}
\end{figure*}


\begin{figure*}[hp]
\begin{center}
\includegraphics[width=\textwidth]{figures/visit_vhelio_compare.pdf}
\end{center}
\caption{%
One point per \emph{visit}, comparing the derived barycentric radial velocity.
Only visits that appear in DR13 and DR14 are plotted here, and I've downsampled
so that an equal number of points are plotted for each SNR bin (there are
$\approx10^5$ in each SNR bin).
In case you can't see the legend, blue points are the higher SNR bin, orange are
lower.
I'm somewhat troubled by the cloud of blue points centered on 0; it seems like
many moderate SNR visits are getting large ($\approx$10's of km/s) velocity
shifts between DR13 and DR14.
}
\end{figure*}


\begin{figure*}[hp]
\begin{center}
\includegraphics[width=\textwidth]{figures/visit_vhelio_vs_snr.pdf}
\end{center}
\caption{%
Another way to look at the last point.
Again, one point per visit, plotting the difference in derived RV normalized by
the reported DR14 uncertainty of the measured RV.
}
\end{figure*}


\begin{figure*}[hp]
\begin{center}
\includegraphics[width=\textwidth]{figures/std_vhelio_compare.pdf}
\end{center}
\caption{%
Now I compute aggregate statistics on all visits for unique
\texttt{APOGEE\_ID}'s as a way to see how the RMS or velocity scatter changes
between DR13 and DR14. Each point here is now a unique \emph{source}, and in
this case represents the standard deviation of all visit velocities for a given
source in DR13 vs. DR14 in linear (left) and log (right).
}
\end{figure*}

\begin{figure*}[hp]
\begin{center}
\includegraphics[width=\textwidth]{figures/MAD_vhelio_compare.pdf}
\end{center}
\caption{%
Same as last figure, but using $\sigma = 1.5 \times {\rm MAD}$, using the median
absolute deviation (MAD) as a more robust estimator of the scatter.
}
\end{figure*}

\begin{figure*}[hp]
\begin{center}
\includegraphics[width=\textwidth]{figures/MAD_vhelio_diff.pdf}
\end{center}
\caption{%
Now I'm plotting the difference in the MAD-estimated scatter vs. median SNR for
each source.
The vertical axis has is log, but linear from $-10^{-3}$ to $10^{-3}$.
The fact that this is symmetric is strange to me.
}
\end{figure*}


\begin{figure}
\begin{tabular}{cc}
\subfloat{\includegraphics[width=2.5in]{figures/0.pdf}} &
\subfloat{\includegraphics[width=2.5in]{figures/1.pdf}} \\
\subfloat{\includegraphics[width=2.5in]{figures/2.pdf}} &
\subfloat{\includegraphics[width=2.5in]{figures/3.pdf}} \\
\subfloat{\includegraphics[width=2.5in]{figures/4.pdf}} &
\subfloat{\includegraphics[width=2.5in]{figures/5.pdf}} \\
\subfloat{\includegraphics[width=2.5in]{figures/6.pdf}} &
\subfloat{\includegraphics[width=2.5in]{figures/7.pdf}} \\
\end{tabular}
\caption{Visit velocities for a random batch of APOGEE stars. Error bars are
plotted for all points, but are sometimes smaller than the marker.}
\end{figure}

\begin{figure}
\begin{tabular}{cc}
\subfloat{\includegraphics[width=2.5in]{figures/8.pdf}} &
\subfloat{\includegraphics[width=2.5in]{figures/9.pdf}} \\
\subfloat{\includegraphics[width=2.5in]{figures/10.pdf}} &
\subfloat{\includegraphics[width=2.5in]{figures/11.pdf}} \\
\subfloat{\includegraphics[width=2.5in]{figures/12.pdf}} &
\subfloat{\includegraphics[width=2.5in]{figures/13.pdf}} \\
\subfloat{\includegraphics[width=2.5in]{figures/14.pdf}} &
\subfloat{\includegraphics[width=2.5in]{figures/15.pdf}} \\
\end{tabular}
\caption{Visit velocities for a random batch of APOGEE stars.}
\end{figure}

\end{document}
